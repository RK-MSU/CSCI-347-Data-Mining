\documentclass[11pt]{article}

    \usepackage[breakable]{tcolorbox}
    \usepackage{parskip} % Stop auto-indenting (to mimic markdown behaviour)
    
    \usepackage{iftex}
    \ifPDFTeX
    	\usepackage[T1]{fontenc}
    	\usepackage{mathpazo}
    \else
    	\usepackage{fontspec}
    \fi

    % Basic figure setup, for now with no caption control since it's done
    % automatically by Pandoc (which extracts ![](path) syntax from Markdown).
    \usepackage{graphicx}
    % Maintain compatibility with old templates. Remove in nbconvert 6.0
    \let\Oldincludegraphics\includegraphics
    % Ensure that by default, figures have no caption (until we provide a
    % proper Figure object with a Caption API and a way to capture that
    % in the conversion process - todo).
    \usepackage{caption}
    \DeclareCaptionFormat{nocaption}{}
    \captionsetup{format=nocaption,aboveskip=0pt,belowskip=0pt}

    \usepackage{float}
    \floatplacement{figure}{H} % forces figures to be placed at the correct location
    \usepackage{xcolor} % Allow colors to be defined
    \usepackage{enumerate} % Needed for markdown enumerations to work
    \usepackage{geometry} % Used to adjust the document margins
    \usepackage{amsmath} % Equations
    \usepackage{amssymb} % Equations
    \usepackage{textcomp} % defines textquotesingle
    % Hack from http://tex.stackexchange.com/a/47451/13684:
    \AtBeginDocument{%
        \def\PYZsq{\textquotesingle}% Upright quotes in Pygmentized code
    }
    \usepackage{upquote} % Upright quotes for verbatim code
    \usepackage{eurosym} % defines \euro
    \usepackage[mathletters]{ucs} % Extended unicode (utf-8) support
    \usepackage{fancyvrb} % verbatim replacement that allows latex
    \usepackage{grffile} % extends the file name processing of package graphics 
                         % to support a larger range
    \makeatletter % fix for old versions of grffile with XeLaTeX
    \@ifpackagelater{grffile}{2019/11/01}
    {
      % Do nothing on new versions
    }
    {
      \def\Gread@@xetex#1{%
        \IfFileExists{"\Gin@base".bb}%
        {\Gread@eps{\Gin@base.bb}}%
        {\Gread@@xetex@aux#1}%
      }
    }
    \makeatother
    \usepackage[Export]{adjustbox} % Used to constrain images to a maximum size
    \adjustboxset{max size={0.9\linewidth}{0.9\paperheight}}

    % The hyperref package gives us a pdf with properly built
    % internal navigation ('pdf bookmarks' for the table of contents,
    % internal cross-reference links, web links for URLs, etc.)
    \usepackage{hyperref}
    % The default LaTeX title has an obnoxious amount of whitespace. By default,
    % titling removes some of it. It also provides customization options.
    \usepackage{titling}
    \usepackage{longtable} % longtable support required by pandoc >1.10
    \usepackage{booktabs}  % table support for pandoc > 1.12.2
    \usepackage[inline]{enumitem} % IRkernel/repr support (it uses the enumerate* environment)
    \usepackage[normalem]{ulem} % ulem is needed to support strikethroughs (\sout)
                                % normalem makes italics be italics, not underlines
    \usepackage{mathrsfs}
    

    
    % Colors for the hyperref package
    \definecolor{urlcolor}{rgb}{0,.145,.698}
    \definecolor{linkcolor}{rgb}{.71,0.21,0.01}
    \definecolor{citecolor}{rgb}{.12,.54,.11}

    % ANSI colors
    \definecolor{ansi-black}{HTML}{3E424D}
    \definecolor{ansi-black-intense}{HTML}{282C36}
    \definecolor{ansi-red}{HTML}{E75C58}
    \definecolor{ansi-red-intense}{HTML}{B22B31}
    \definecolor{ansi-green}{HTML}{00A250}
    \definecolor{ansi-green-intense}{HTML}{007427}
    \definecolor{ansi-yellow}{HTML}{DDB62B}
    \definecolor{ansi-yellow-intense}{HTML}{B27D12}
    \definecolor{ansi-blue}{HTML}{208FFB}
    \definecolor{ansi-blue-intense}{HTML}{0065CA}
    \definecolor{ansi-magenta}{HTML}{D160C4}
    \definecolor{ansi-magenta-intense}{HTML}{A03196}
    \definecolor{ansi-cyan}{HTML}{60C6C8}
    \definecolor{ansi-cyan-intense}{HTML}{258F8F}
    \definecolor{ansi-white}{HTML}{C5C1B4}
    \definecolor{ansi-white-intense}{HTML}{A1A6B2}
    \definecolor{ansi-default-inverse-fg}{HTML}{FFFFFF}
    \definecolor{ansi-default-inverse-bg}{HTML}{000000}

    % common color for the border for error outputs.
    \definecolor{outerrorbackground}{HTML}{FFDFDF}

    % commands and environments needed by pandoc snippets
    % extracted from the output of `pandoc -s`
    \providecommand{\tightlist}{%
      \setlength{\itemsep}{0pt}\setlength{\parskip}{0pt}}
    \DefineVerbatimEnvironment{Highlighting}{Verbatim}{commandchars=\\\{\}}
    % Add ',fontsize=\small' for more characters per line
    \newenvironment{Shaded}{}{}
    \newcommand{\KeywordTok}[1]{\textcolor[rgb]{0.00,0.44,0.13}{\textbf{{#1}}}}
    \newcommand{\DataTypeTok}[1]{\textcolor[rgb]{0.56,0.13,0.00}{{#1}}}
    \newcommand{\DecValTok}[1]{\textcolor[rgb]{0.25,0.63,0.44}{{#1}}}
    \newcommand{\BaseNTok}[1]{\textcolor[rgb]{0.25,0.63,0.44}{{#1}}}
    \newcommand{\FloatTok}[1]{\textcolor[rgb]{0.25,0.63,0.44}{{#1}}}
    \newcommand{\CharTok}[1]{\textcolor[rgb]{0.25,0.44,0.63}{{#1}}}
    \newcommand{\StringTok}[1]{\textcolor[rgb]{0.25,0.44,0.63}{{#1}}}
    \newcommand{\CommentTok}[1]{\textcolor[rgb]{0.38,0.63,0.69}{\textit{{#1}}}}
    \newcommand{\OtherTok}[1]{\textcolor[rgb]{0.00,0.44,0.13}{{#1}}}
    \newcommand{\AlertTok}[1]{\textcolor[rgb]{1.00,0.00,0.00}{\textbf{{#1}}}}
    \newcommand{\FunctionTok}[1]{\textcolor[rgb]{0.02,0.16,0.49}{{#1}}}
    \newcommand{\RegionMarkerTok}[1]{{#1}}
    \newcommand{\ErrorTok}[1]{\textcolor[rgb]{1.00,0.00,0.00}{\textbf{{#1}}}}
    \newcommand{\NormalTok}[1]{{#1}}
    
    % Additional commands for more recent versions of Pandoc
    \newcommand{\ConstantTok}[1]{\textcolor[rgb]{0.53,0.00,0.00}{{#1}}}
    \newcommand{\SpecialCharTok}[1]{\textcolor[rgb]{0.25,0.44,0.63}{{#1}}}
    \newcommand{\VerbatimStringTok}[1]{\textcolor[rgb]{0.25,0.44,0.63}{{#1}}}
    \newcommand{\SpecialStringTok}[1]{\textcolor[rgb]{0.73,0.40,0.53}{{#1}}}
    \newcommand{\ImportTok}[1]{{#1}}
    \newcommand{\DocumentationTok}[1]{\textcolor[rgb]{0.73,0.13,0.13}{\textit{{#1}}}}
    \newcommand{\AnnotationTok}[1]{\textcolor[rgb]{0.38,0.63,0.69}{\textbf{\textit{{#1}}}}}
    \newcommand{\CommentVarTok}[1]{\textcolor[rgb]{0.38,0.63,0.69}{\textbf{\textit{{#1}}}}}
    \newcommand{\VariableTok}[1]{\textcolor[rgb]{0.10,0.09,0.49}{{#1}}}
    \newcommand{\ControlFlowTok}[1]{\textcolor[rgb]{0.00,0.44,0.13}{\textbf{{#1}}}}
    \newcommand{\OperatorTok}[1]{\textcolor[rgb]{0.40,0.40,0.40}{{#1}}}
    \newcommand{\BuiltInTok}[1]{{#1}}
    \newcommand{\ExtensionTok}[1]{{#1}}
    \newcommand{\PreprocessorTok}[1]{\textcolor[rgb]{0.74,0.48,0.00}{{#1}}}
    \newcommand{\AttributeTok}[1]{\textcolor[rgb]{0.49,0.56,0.16}{{#1}}}
    \newcommand{\InformationTok}[1]{\textcolor[rgb]{0.38,0.63,0.69}{\textbf{\textit{{#1}}}}}
    \newcommand{\WarningTok}[1]{\textcolor[rgb]{0.38,0.63,0.69}{\textbf{\textit{{#1}}}}}
    
    
    % Define a nice break command that doesn't care if a line doesn't already
    % exist.
    \def\br{\hspace*{\fill} \\* }
    % Math Jax compatibility definitions
    \def\gt{>}
    \def\lt{<}
    \let\Oldtex\TeX
    \let\Oldlatex\LaTeX
    \renewcommand{\TeX}{\textrm{\Oldtex}}
    \renewcommand{\LaTeX}{\textrm{\Oldlatex}}
    % Document parameters
    % Document title
    \title{Project 02}
    
    
    
    
    
% Pygments definitions
\makeatletter
\def\PY@reset{\let\PY@it=\relax \let\PY@bf=\relax%
    \let\PY@ul=\relax \let\PY@tc=\relax%
    \let\PY@bc=\relax \let\PY@ff=\relax}
\def\PY@tok#1{\csname PY@tok@#1\endcsname}
\def\PY@toks#1+{\ifx\relax#1\empty\else%
    \PY@tok{#1}\expandafter\PY@toks\fi}
\def\PY@do#1{\PY@bc{\PY@tc{\PY@ul{%
    \PY@it{\PY@bf{\PY@ff{#1}}}}}}}
\def\PY#1#2{\PY@reset\PY@toks#1+\relax+\PY@do{#2}}

\@namedef{PY@tok@w}{\def\PY@tc##1{\textcolor[rgb]{0.73,0.73,0.73}{##1}}}
\@namedef{PY@tok@c}{\let\PY@it=\textit\def\PY@tc##1{\textcolor[rgb]{0.24,0.48,0.48}{##1}}}
\@namedef{PY@tok@cp}{\def\PY@tc##1{\textcolor[rgb]{0.61,0.40,0.00}{##1}}}
\@namedef{PY@tok@k}{\let\PY@bf=\textbf\def\PY@tc##1{\textcolor[rgb]{0.00,0.50,0.00}{##1}}}
\@namedef{PY@tok@kp}{\def\PY@tc##1{\textcolor[rgb]{0.00,0.50,0.00}{##1}}}
\@namedef{PY@tok@kt}{\def\PY@tc##1{\textcolor[rgb]{0.69,0.00,0.25}{##1}}}
\@namedef{PY@tok@o}{\def\PY@tc##1{\textcolor[rgb]{0.40,0.40,0.40}{##1}}}
\@namedef{PY@tok@ow}{\let\PY@bf=\textbf\def\PY@tc##1{\textcolor[rgb]{0.67,0.13,1.00}{##1}}}
\@namedef{PY@tok@nb}{\def\PY@tc##1{\textcolor[rgb]{0.00,0.50,0.00}{##1}}}
\@namedef{PY@tok@nf}{\def\PY@tc##1{\textcolor[rgb]{0.00,0.00,1.00}{##1}}}
\@namedef{PY@tok@nc}{\let\PY@bf=\textbf\def\PY@tc##1{\textcolor[rgb]{0.00,0.00,1.00}{##1}}}
\@namedef{PY@tok@nn}{\let\PY@bf=\textbf\def\PY@tc##1{\textcolor[rgb]{0.00,0.00,1.00}{##1}}}
\@namedef{PY@tok@ne}{\let\PY@bf=\textbf\def\PY@tc##1{\textcolor[rgb]{0.80,0.25,0.22}{##1}}}
\@namedef{PY@tok@nv}{\def\PY@tc##1{\textcolor[rgb]{0.10,0.09,0.49}{##1}}}
\@namedef{PY@tok@no}{\def\PY@tc##1{\textcolor[rgb]{0.53,0.00,0.00}{##1}}}
\@namedef{PY@tok@nl}{\def\PY@tc##1{\textcolor[rgb]{0.46,0.46,0.00}{##1}}}
\@namedef{PY@tok@ni}{\let\PY@bf=\textbf\def\PY@tc##1{\textcolor[rgb]{0.44,0.44,0.44}{##1}}}
\@namedef{PY@tok@na}{\def\PY@tc##1{\textcolor[rgb]{0.41,0.47,0.13}{##1}}}
\@namedef{PY@tok@nt}{\let\PY@bf=\textbf\def\PY@tc##1{\textcolor[rgb]{0.00,0.50,0.00}{##1}}}
\@namedef{PY@tok@nd}{\def\PY@tc##1{\textcolor[rgb]{0.67,0.13,1.00}{##1}}}
\@namedef{PY@tok@s}{\def\PY@tc##1{\textcolor[rgb]{0.73,0.13,0.13}{##1}}}
\@namedef{PY@tok@sd}{\let\PY@it=\textit\def\PY@tc##1{\textcolor[rgb]{0.73,0.13,0.13}{##1}}}
\@namedef{PY@tok@si}{\let\PY@bf=\textbf\def\PY@tc##1{\textcolor[rgb]{0.64,0.35,0.47}{##1}}}
\@namedef{PY@tok@se}{\let\PY@bf=\textbf\def\PY@tc##1{\textcolor[rgb]{0.67,0.36,0.12}{##1}}}
\@namedef{PY@tok@sr}{\def\PY@tc##1{\textcolor[rgb]{0.64,0.35,0.47}{##1}}}
\@namedef{PY@tok@ss}{\def\PY@tc##1{\textcolor[rgb]{0.10,0.09,0.49}{##1}}}
\@namedef{PY@tok@sx}{\def\PY@tc##1{\textcolor[rgb]{0.00,0.50,0.00}{##1}}}
\@namedef{PY@tok@m}{\def\PY@tc##1{\textcolor[rgb]{0.40,0.40,0.40}{##1}}}
\@namedef{PY@tok@gh}{\let\PY@bf=\textbf\def\PY@tc##1{\textcolor[rgb]{0.00,0.00,0.50}{##1}}}
\@namedef{PY@tok@gu}{\let\PY@bf=\textbf\def\PY@tc##1{\textcolor[rgb]{0.50,0.00,0.50}{##1}}}
\@namedef{PY@tok@gd}{\def\PY@tc##1{\textcolor[rgb]{0.63,0.00,0.00}{##1}}}
\@namedef{PY@tok@gi}{\def\PY@tc##1{\textcolor[rgb]{0.00,0.52,0.00}{##1}}}
\@namedef{PY@tok@gr}{\def\PY@tc##1{\textcolor[rgb]{0.89,0.00,0.00}{##1}}}
\@namedef{PY@tok@ge}{\let\PY@it=\textit}
\@namedef{PY@tok@gs}{\let\PY@bf=\textbf}
\@namedef{PY@tok@gp}{\let\PY@bf=\textbf\def\PY@tc##1{\textcolor[rgb]{0.00,0.00,0.50}{##1}}}
\@namedef{PY@tok@go}{\def\PY@tc##1{\textcolor[rgb]{0.44,0.44,0.44}{##1}}}
\@namedef{PY@tok@gt}{\def\PY@tc##1{\textcolor[rgb]{0.00,0.27,0.87}{##1}}}
\@namedef{PY@tok@err}{\def\PY@bc##1{{\setlength{\fboxsep}{\string -\fboxrule}\fcolorbox[rgb]{1.00,0.00,0.00}{1,1,1}{\strut ##1}}}}
\@namedef{PY@tok@kc}{\let\PY@bf=\textbf\def\PY@tc##1{\textcolor[rgb]{0.00,0.50,0.00}{##1}}}
\@namedef{PY@tok@kd}{\let\PY@bf=\textbf\def\PY@tc##1{\textcolor[rgb]{0.00,0.50,0.00}{##1}}}
\@namedef{PY@tok@kn}{\let\PY@bf=\textbf\def\PY@tc##1{\textcolor[rgb]{0.00,0.50,0.00}{##1}}}
\@namedef{PY@tok@kr}{\let\PY@bf=\textbf\def\PY@tc##1{\textcolor[rgb]{0.00,0.50,0.00}{##1}}}
\@namedef{PY@tok@bp}{\def\PY@tc##1{\textcolor[rgb]{0.00,0.50,0.00}{##1}}}
\@namedef{PY@tok@fm}{\def\PY@tc##1{\textcolor[rgb]{0.00,0.00,1.00}{##1}}}
\@namedef{PY@tok@vc}{\def\PY@tc##1{\textcolor[rgb]{0.10,0.09,0.49}{##1}}}
\@namedef{PY@tok@vg}{\def\PY@tc##1{\textcolor[rgb]{0.10,0.09,0.49}{##1}}}
\@namedef{PY@tok@vi}{\def\PY@tc##1{\textcolor[rgb]{0.10,0.09,0.49}{##1}}}
\@namedef{PY@tok@vm}{\def\PY@tc##1{\textcolor[rgb]{0.10,0.09,0.49}{##1}}}
\@namedef{PY@tok@sa}{\def\PY@tc##1{\textcolor[rgb]{0.73,0.13,0.13}{##1}}}
\@namedef{PY@tok@sb}{\def\PY@tc##1{\textcolor[rgb]{0.73,0.13,0.13}{##1}}}
\@namedef{PY@tok@sc}{\def\PY@tc##1{\textcolor[rgb]{0.73,0.13,0.13}{##1}}}
\@namedef{PY@tok@dl}{\def\PY@tc##1{\textcolor[rgb]{0.73,0.13,0.13}{##1}}}
\@namedef{PY@tok@s2}{\def\PY@tc##1{\textcolor[rgb]{0.73,0.13,0.13}{##1}}}
\@namedef{PY@tok@sh}{\def\PY@tc##1{\textcolor[rgb]{0.73,0.13,0.13}{##1}}}
\@namedef{PY@tok@s1}{\def\PY@tc##1{\textcolor[rgb]{0.73,0.13,0.13}{##1}}}
\@namedef{PY@tok@mb}{\def\PY@tc##1{\textcolor[rgb]{0.40,0.40,0.40}{##1}}}
\@namedef{PY@tok@mf}{\def\PY@tc##1{\textcolor[rgb]{0.40,0.40,0.40}{##1}}}
\@namedef{PY@tok@mh}{\def\PY@tc##1{\textcolor[rgb]{0.40,0.40,0.40}{##1}}}
\@namedef{PY@tok@mi}{\def\PY@tc##1{\textcolor[rgb]{0.40,0.40,0.40}{##1}}}
\@namedef{PY@tok@il}{\def\PY@tc##1{\textcolor[rgb]{0.40,0.40,0.40}{##1}}}
\@namedef{PY@tok@mo}{\def\PY@tc##1{\textcolor[rgb]{0.40,0.40,0.40}{##1}}}
\@namedef{PY@tok@ch}{\let\PY@it=\textit\def\PY@tc##1{\textcolor[rgb]{0.24,0.48,0.48}{##1}}}
\@namedef{PY@tok@cm}{\let\PY@it=\textit\def\PY@tc##1{\textcolor[rgb]{0.24,0.48,0.48}{##1}}}
\@namedef{PY@tok@cpf}{\let\PY@it=\textit\def\PY@tc##1{\textcolor[rgb]{0.24,0.48,0.48}{##1}}}
\@namedef{PY@tok@c1}{\let\PY@it=\textit\def\PY@tc##1{\textcolor[rgb]{0.24,0.48,0.48}{##1}}}
\@namedef{PY@tok@cs}{\let\PY@it=\textit\def\PY@tc##1{\textcolor[rgb]{0.24,0.48,0.48}{##1}}}

\def\PYZbs{\char`\\}
\def\PYZus{\char`\_}
\def\PYZob{\char`\{}
\def\PYZcb{\char`\}}
\def\PYZca{\char`\^}
\def\PYZam{\char`\&}
\def\PYZlt{\char`\<}
\def\PYZgt{\char`\>}
\def\PYZsh{\char`\#}
\def\PYZpc{\char`\%}
\def\PYZdl{\char`\$}
\def\PYZhy{\char`\-}
\def\PYZsq{\char`\'}
\def\PYZdq{\char`\"}
\def\PYZti{\char`\~}
% for compatibility with earlier versions
\def\PYZat{@}
\def\PYZlb{[}
\def\PYZrb{]}
\makeatother


    % For linebreaks inside Verbatim environment from package fancyvrb. 
    \makeatletter
        \newbox\Wrappedcontinuationbox 
        \newbox\Wrappedvisiblespacebox 
        \newcommand*\Wrappedvisiblespace {\textcolor{red}{\textvisiblespace}} 
        \newcommand*\Wrappedcontinuationsymbol {\textcolor{red}{\llap{\tiny$\m@th\hookrightarrow$}}} 
        \newcommand*\Wrappedcontinuationindent {3ex } 
        \newcommand*\Wrappedafterbreak {\kern\Wrappedcontinuationindent\copy\Wrappedcontinuationbox} 
        % Take advantage of the already applied Pygments mark-up to insert 
        % potential linebreaks for TeX processing. 
        %        {, <, #, %, $, ' and ": go to next line. 
        %        _, }, ^, &, >, - and ~: stay at end of broken line. 
        % Use of \textquotesingle for straight quote. 
        \newcommand*\Wrappedbreaksatspecials {% 
            \def\PYGZus{\discretionary{\char`\_}{\Wrappedafterbreak}{\char`\_}}% 
            \def\PYGZob{\discretionary{}{\Wrappedafterbreak\char`\{}{\char`\{}}% 
            \def\PYGZcb{\discretionary{\char`\}}{\Wrappedafterbreak}{\char`\}}}% 
            \def\PYGZca{\discretionary{\char`\^}{\Wrappedafterbreak}{\char`\^}}% 
            \def\PYGZam{\discretionary{\char`\&}{\Wrappedafterbreak}{\char`\&}}% 
            \def\PYGZlt{\discretionary{}{\Wrappedafterbreak\char`\<}{\char`\<}}% 
            \def\PYGZgt{\discretionary{\char`\>}{\Wrappedafterbreak}{\char`\>}}% 
            \def\PYGZsh{\discretionary{}{\Wrappedafterbreak\char`\#}{\char`\#}}% 
            \def\PYGZpc{\discretionary{}{\Wrappedafterbreak\char`\%}{\char`\%}}% 
            \def\PYGZdl{\discretionary{}{\Wrappedafterbreak\char`\$}{\char`\$}}% 
            \def\PYGZhy{\discretionary{\char`\-}{\Wrappedafterbreak}{\char`\-}}% 
            \def\PYGZsq{\discretionary{}{\Wrappedafterbreak\textquotesingle}{\textquotesingle}}% 
            \def\PYGZdq{\discretionary{}{\Wrappedafterbreak\char`\"}{\char`\"}}% 
            \def\PYGZti{\discretionary{\char`\~}{\Wrappedafterbreak}{\char`\~}}% 
        } 
        % Some characters . , ; ? ! / are not pygmentized. 
        % This macro makes them "active" and they will insert potential linebreaks 
        \newcommand*\Wrappedbreaksatpunct {% 
            \lccode`\~`\.\lowercase{\def~}{\discretionary{\hbox{\char`\.}}{\Wrappedafterbreak}{\hbox{\char`\.}}}% 
            \lccode`\~`\,\lowercase{\def~}{\discretionary{\hbox{\char`\,}}{\Wrappedafterbreak}{\hbox{\char`\,}}}% 
            \lccode`\~`\;\lowercase{\def~}{\discretionary{\hbox{\char`\;}}{\Wrappedafterbreak}{\hbox{\char`\;}}}% 
            \lccode`\~`\:\lowercase{\def~}{\discretionary{\hbox{\char`\:}}{\Wrappedafterbreak}{\hbox{\char`\:}}}% 
            \lccode`\~`\?\lowercase{\def~}{\discretionary{\hbox{\char`\?}}{\Wrappedafterbreak}{\hbox{\char`\?}}}% 
            \lccode`\~`\!\lowercase{\def~}{\discretionary{\hbox{\char`\!}}{\Wrappedafterbreak}{\hbox{\char`\!}}}% 
            \lccode`\~`\/\lowercase{\def~}{\discretionary{\hbox{\char`\/}}{\Wrappedafterbreak}{\hbox{\char`\/}}}% 
            \catcode`\.\active
            \catcode`\,\active 
            \catcode`\;\active
            \catcode`\:\active
            \catcode`\?\active
            \catcode`\!\active
            \catcode`\/\active 
            \lccode`\~`\~ 	
        }
    \makeatother

    \let\OriginalVerbatim=\Verbatim
    \makeatletter
    \renewcommand{\Verbatim}[1][1]{%
        %\parskip\z@skip
        \sbox\Wrappedcontinuationbox {\Wrappedcontinuationsymbol}%
        \sbox\Wrappedvisiblespacebox {\FV@SetupFont\Wrappedvisiblespace}%
        \def\FancyVerbFormatLine ##1{\hsize\linewidth
            \vtop{\raggedright\hyphenpenalty\z@\exhyphenpenalty\z@
                \doublehyphendemerits\z@\finalhyphendemerits\z@
                \strut ##1\strut}%
        }%
        % If the linebreak is at a space, the latter will be displayed as visible
        % space at end of first line, and a continuation symbol starts next line.
        % Stretch/shrink are however usually zero for typewriter font.
        \def\FV@Space {%
            \nobreak\hskip\z@ plus\fontdimen3\font minus\fontdimen4\font
            \discretionary{\copy\Wrappedvisiblespacebox}{\Wrappedafterbreak}
            {\kern\fontdimen2\font}%
        }%
        
        % Allow breaks at special characters using \PYG... macros.
        \Wrappedbreaksatspecials
        % Breaks at punctuation characters . , ; ? ! and / need catcode=\active 	
        \OriginalVerbatim[#1,codes*=\Wrappedbreaksatpunct]%
    }
    \makeatother

    % Exact colors from NB
    \definecolor{incolor}{HTML}{303F9F}
    \definecolor{outcolor}{HTML}{D84315}
    \definecolor{cellborder}{HTML}{CFCFCF}
    \definecolor{cellbackground}{HTML}{F7F7F7}
    
    % prompt
    \makeatletter
    \newcommand{\boxspacing}{\kern\kvtcb@left@rule\kern\kvtcb@boxsep}
    \makeatother
    \newcommand{\prompt}[4]{
        {\ttfamily\llap{{\color{#2}[#3]:\hspace{3pt}#4}}\vspace{-\baselineskip}}
    }
    

    
    % Prevent overflowing lines due to hard-to-break entities
    \sloppy 
    % Setup hyperref package
    \hypersetup{
      breaklinks=true,  % so long urls are correctly broken across lines
      colorlinks=true,
      urlcolor=urlcolor,
      linkcolor=linkcolor,
      citecolor=citecolor,
      }
    % Slightly bigger margins than the latex defaults
    
    \geometry{verbose,tmargin=1in,bmargin=1in,lmargin=1in,rmargin=1in}
    
    

\begin{document}
    
    \maketitle
    
    

    
    River Kelly \& Kyler Gappa

CSCI-347: Data Mining

Project 02: Exploring Graph Data

    Partner work is allowed on this project.

Choose a data set that you are interested in from one of the following
sources: - SNAP collection: https://snap.stanford.edu/data - Network
Repository: http://networkrepository.com/index.php

Run all the analysis in this project on the largest connected component
of the graph. Note that many of these datasets are quite large. If
analyzing the data its taking too long, you may pre-process it by taking
a sample of the graph first, and then extracting the largest connected
component, to get the graph down to a manageable size.

    \hypertarget{python-setup-code}{%
\section*{Python Setup Code}\label{python-setup-code}}

    \begin{tcolorbox}[breakable, size=fbox, boxrule=1pt, pad at break*=1mm,colback=cellbackground, colframe=cellborder]
\prompt{In}{incolor}{ }{\boxspacing}
\begin{Verbatim}[commandchars=\\\{\}]
\PY{c+c1}{\PYZsh{} import libraries}
\PY{k+kn}{import} \PY{n+nn}{numpy} \PY{k}{as} \PY{n+nn}{np}
\PY{k+kn}{import} \PY{n+nn}{pandas} \PY{k}{as} \PY{n+nn}{pd}
\PY{k+kn}{import} \PY{n+nn}{networkx} \PY{k}{as} \PY{n+nn}{nx}
\PY{k+kn}{import} \PY{n+nn}{matplotlib}\PY{n+nn}{.}\PY{n+nn}{pyplot} \PY{k}{as} \PY{n+nn}{plt}
\PY{k+kn}{import} \PY{n+nn}{urllib}\PY{n+nn}{.}\PY{n+nn}{request}
\PY{k+kn}{import} \PY{n+nn}{io}
\PY{k+kn}{import} \PY{n+nn}{gzip}
\PY{k+kn}{import} \PY{n+nn}{random} \PY{k}{as} \PY{n+nn}{rd}
\end{Verbatim}
\end{tcolorbox}

    \begin{tcolorbox}[breakable, size=fbox, boxrule=1pt, pad at break*=1mm,colback=cellbackground, colframe=cellborder]
\prompt{In}{incolor}{ }{\boxspacing}
\begin{Verbatim}[commandchars=\\\{\}]
\PY{n}{DATA\PYZus{}URL} \PY{o}{=} \PY{l+s+s1}{\PYZsq{}}\PY{l+s+s1}{https://snap.stanford.edu/data/facebook\PYZus{}combined.txt.gz}\PY{l+s+s1}{\PYZsq{}}
\end{Verbatim}
\end{tcolorbox}

    \begin{tcolorbox}[breakable, size=fbox, boxrule=1pt, pad at break*=1mm,colback=cellbackground, colframe=cellborder]
\prompt{In}{incolor}{ }{\boxspacing}
\begin{Verbatim}[commandchars=\\\{\}]
\PY{k}{def} \PY{n+nf}{getFileData}\PY{p}{(}\PY{n}{url}\PY{p}{:} \PY{n+nb}{str}\PY{p}{)} \PY{o}{\PYZhy{}}\PY{o}{\PYZgt{}} \PY{n+nb}{list}\PY{p}{:}
    \PY{n}{response} \PY{o}{=} \PY{n}{urllib}\PY{o}{.}\PY{n}{request}\PY{o}{.}\PY{n}{urlopen}\PY{p}{(}\PY{n}{url}\PY{p}{)}
    \PY{n}{compressed\PYZus{}file} \PY{o}{=} \PY{n}{io}\PY{o}{.}\PY{n}{BytesIO}\PY{p}{(}\PY{n}{response}\PY{o}{.}\PY{n}{read}\PY{p}{(}\PY{p}{)}\PY{p}{)}
    \PY{n}{decompressed\PYZus{}file} \PY{o}{=} \PY{n}{gzip}\PY{o}{.}\PY{n}{GzipFile}\PY{p}{(}\PY{n}{fileobj}\PY{o}{=}\PY{n}{compressed\PYZus{}file}\PY{p}{)}
    \PY{c+c1}{\PYZsh{} read file to edges list}
    \PY{n}{edges}\PY{p}{:} \PY{n+nb}{list} \PY{o}{=} \PY{n+nb}{list}\PY{p}{(}\PY{p}{)}
    \PY{k}{while} \PY{k+kc}{True}\PY{p}{:}
        \PY{n}{line} \PY{o}{=} \PY{n}{decompressed\PYZus{}file}\PY{o}{.}\PY{n}{readline}\PY{p}{(}\PY{p}{)}
        \PY{k}{if} \PY{o+ow}{not} \PY{n}{line}\PY{p}{:} \PY{k}{break} \PY{c+c1}{\PYZsh{} no more lines to read}
        \PY{c+c1}{\PYZsh{} parse line string}
        \PY{n}{line} \PY{o}{=} \PY{n+nb}{str}\PY{p}{(}\PY{n}{line}\PY{o}{.}\PY{n}{decode}\PY{p}{(}\PY{l+s+s2}{\PYZdq{}}\PY{l+s+s2}{utf\PYZhy{}8}\PY{l+s+s2}{\PYZdq{}}\PY{p}{)}\PY{p}{)}\PY{o}{.}\PY{n}{strip}\PY{p}{(}\PY{p}{)}
        \PY{n}{edge\PYZus{}str\PYZus{}data} \PY{o}{=} \PY{n}{line}\PY{o}{.}\PY{n}{split}\PY{p}{(}\PY{l+s+s1}{\PYZsq{}}\PY{l+s+s1}{ }\PY{l+s+s1}{\PYZsq{}}\PY{p}{)}
        \PY{n}{point\PYZus{}1} \PY{o}{=} \PY{n+nb}{int}\PY{p}{(}\PY{n}{edge\PYZus{}str\PYZus{}data}\PY{p}{[}\PY{l+m+mi}{0}\PY{p}{]}\PY{p}{)}
        \PY{n}{point\PYZus{}2} \PY{o}{=} \PY{n+nb}{int}\PY{p}{(}\PY{n}{edge\PYZus{}str\PYZus{}data}\PY{p}{[}\PY{l+m+mi}{1}\PY{p}{]}\PY{p}{)}
        \PY{n}{edge} \PY{o}{=} \PY{p}{(}\PY{n}{point\PYZus{}1}\PY{p}{,} \PY{n}{point\PYZus{}2}\PY{p}{)}
        \PY{n}{edges}\PY{o}{.}\PY{n}{append}\PY{p}{(}\PY{n}{edge}\PY{p}{)}
    \PY{n}{decompressed\PYZus{}file}\PY{o}{.}\PY{n}{close}\PY{p}{(}\PY{p}{)}
    \PY{n}{compressed\PYZus{}file}\PY{o}{.}\PY{n}{close}\PY{p}{(}\PY{p}{)}
    \PY{k}{return} \PY{n}{edges}
\end{Verbatim}
\end{tcolorbox}

    \hypertarget{part-1-think-about-the-data}{%
\section*{Part 1: Think about the
data}\label{part-1-think-about-the-data}}

    This data set is interesting because it shows the connectivity of each
person social circle. This can lead to a better understanding of how
people with different interests connect to each other. We expect that
nodes with a high level of centrality to be people with a large amount
of popular interests. These people could then be good to advertise
around as they are most likely to influence a large population. We did
not take a sample of the elements from the data set.

    \hypertarget{part-2-write-python-code-for-graph-analysis}{%
\section*{Part 2: Write Python code for graph
analysis}\label{part-2-write-python-code-for-graph-analysis}}

    Write the following functions in Python. You may assume that the input
graph is unweighted, undirected, and simple -- has no parallel edges and
no loops. Functions provided by networkx can be used within your code,
as long as the function does not perform the same task as what you are
being asked to implement. For example, you cannot use networkx's
betweenness centrality function within your own betweenness centrality
function, but you can use networkx's functions for finding shortest
paths. You may also assume that vertices are represented as integers (so
the pair (1,3) indicates that there is an edge between vertex 1 and 3,
for example).

    \hypertarget{points-number-of-vertices}{%
\subsection*{1. (5 points) Number of
vertices}\label{points-number-of-vertices}}

    A function that takes the following input: a list of edges representing
a graph, where each edge is a pair. The output should be the number of
vertices.

    \begin{tcolorbox}[breakable, size=fbox, boxrule=1pt, pad at break*=1mm,colback=cellbackground, colframe=cellborder]
\prompt{In}{incolor}{ }{\boxspacing}
\begin{Verbatim}[commandchars=\\\{\}]
\PY{c+c1}{\PYZsh{} numberOfVertices() \PYZhy{} returns the number of vertices}
\PY{c+c1}{\PYZsh{} params: }
\PY{c+c1}{\PYZsh{}   \PYZhy{} edges: list}
\PY{k}{def} \PY{n+nf}{numberOfVertices}\PY{p}{(}\PY{n}{edges}\PY{p}{:} \PY{n+nb}{list}\PY{p}{)}\PY{p}{:}
    \PY{c+c1}{\PYZsh{} list to maintain the individual vertices}
    \PY{n}{vertices\PYZus{}list} \PY{o}{=} \PY{n+nb}{list}\PY{p}{(}\PY{p}{)}
    \PY{c+c1}{\PYZsh{} loop through list of edges}
    \PY{k}{for} \PY{n}{edge} \PY{o+ow}{in} \PY{n}{edges}\PY{p}{:}
        \PY{c+c1}{\PYZsh{} loop through each vertex in the edge}
        \PY{k}{for} \PY{n}{vertex} \PY{o+ow}{in} \PY{n}{edge}\PY{p}{:}
            \PY{c+c1}{\PYZsh{} check if vertex has already been accounted for in the list}
            \PY{k}{if} \PY{n}{vertex} \PY{o+ow}{not} \PY{o+ow}{in} \PY{n}{vertices\PYZus{}list}\PY{p}{:}
                \PY{c+c1}{\PYZsh{} add the vertex to the list of all vertices}
                \PY{n}{vertices\PYZus{}list}\PY{o}{.}\PY{n}{append}\PY{p}{(}\PY{n}{vertex}\PY{p}{)}
    \PY{c+c1}{\PYZsh{} return the length of the vertices list (i.e. the total number of vertices)}
    \PY{k}{return} \PY{n+nb}{len}\PY{p}{(}\PY{n}{vertices\PYZus{}list}\PY{p}{)}
\end{Verbatim}
\end{tcolorbox}

    \hypertarget{points-degree-of-a-vertex}{%
\subsection*{2. (5 points) Degree of a
vertex}\label{points-degree-of-a-vertex}}

    A function that takes the following input: a list of edges representing
a graph, where each edge is a pair, and a vertex index that is an
integer. The output should be the degree of the input vertex.

    \begin{tcolorbox}[breakable, size=fbox, boxrule=1pt, pad at break*=1mm,colback=cellbackground, colframe=cellborder]
\prompt{In}{incolor}{ }{\boxspacing}
\begin{Verbatim}[commandchars=\\\{\}]
\PY{c+c1}{\PYZsh{} vertexDegree() \PYZhy{} Computes the degree of a given vertex}
\PY{c+c1}{\PYZsh{} params:}
\PY{c+c1}{\PYZsh{}   \PYZhy{} edges: list }
\PY{c+c1}{\PYZsh{}   \PYZhy{} vertex: int}
\PY{k}{def} \PY{n+nf}{vertexDegree}\PY{p}{(}\PY{n}{edges}\PY{p}{:} \PY{n+nb}{list}\PY{p}{,} \PY{n}{vertex}\PY{p}{:} \PY{n+nb}{int}\PY{p}{,} \PY{n}{returnAdjacentVertexList}\PY{p}{:} \PY{n+nb}{bool} \PY{o}{=} \PY{k+kc}{False}\PY{p}{)}\PY{p}{:}
    \PY{c+c1}{\PYZsh{} list of adjacent verticies}
    \PY{n}{adjacent\PYZus{}vertices\PYZus{}list} \PY{o}{=} \PY{n+nb}{list}\PY{p}{(}\PY{p}{)}
    \PY{c+c1}{\PYZsh{} loop through the list of edges (i.e list of vertice }
    \PY{c+c1}{\PYZsh{} pairs that make up the edges)}
    \PY{k}{for} \PY{n}{edge} \PY{o+ow}{in} \PY{n}{edges}\PY{p}{:}
        \PY{c+c1}{\PYZsh{} does the edge contain the vertex of interest}
        \PY{k}{if} \PY{n}{vertex} \PY{o+ow}{not} \PY{o+ow}{in} \PY{n}{edge}\PY{p}{:}
            \PY{c+c1}{\PYZsh{} skip this edge, not vertices of interes}
            \PY{k}{continue}
        \PY{c+c1}{\PYZsh{} this edge does contain a vertice of interest.}
        \PY{c+c1}{\PYZsh{} lets loop through each of the vertices that make up the}
        \PY{c+c1}{\PYZsh{} edge and see if we need to count it}
        \PY{k}{for} \PY{n}{edge\PYZus{}vertex} \PY{o+ow}{in} \PY{n}{edge}\PY{p}{:}
            \PY{c+c1}{\PYZsh{} is this edge\PYZus{}vertex the vertex of interest}
            \PY{k}{if} \PY{n}{edge\PYZus{}vertex} \PY{o}{==} \PY{n}{vertex}\PY{p}{:}
                \PY{c+c1}{\PYZsh{} it is, so lets skip it}
                \PY{k}{continue}
            \PY{c+c1}{\PYZsh{} have we already counted this vertice}
            \PY{k}{if} \PY{n}{edge\PYZus{}vertex} \PY{o+ow}{in} \PY{n}{adjacent\PYZus{}vertices\PYZus{}list}\PY{p}{:}
                \PY{c+c1}{\PYZsh{} we have, so lets skip it}
                \PY{k}{continue}
            \PY{c+c1}{\PYZsh{} append the vertice to the list of adjacent vertices}
            \PY{n}{adjacent\PYZus{}vertices\PYZus{}list}\PY{o}{.}\PY{n}{append}\PY{p}{(}\PY{n}{edge\PYZus{}vertex}\PY{p}{)}
    \PY{k}{if} \PY{n}{returnAdjacentVertexList} \PY{o+ow}{is} \PY{k+kc}{True}\PY{p}{:}
        \PY{k}{return} \PY{n}{adjacent\PYZus{}vertices\PYZus{}list}
    \PY{k}{return} \PY{n+nb}{len}\PY{p}{(}\PY{n}{adjacent\PYZus{}vertices\PYZus{}list}\PY{p}{)}
\end{Verbatim}
\end{tcolorbox}

    \hypertarget{points-clustering-coefficient-of-a-vertex}{%
\subsection*{3. (5 points) Clustering coefficient of a
vertex}\label{points-clustering-coefficient-of-a-vertex}}

    A function that takes the following input: a list of edges representing
a graph, where each edge is a pair, and a vertex index that is an
integer. The output should be the clustering coefficient of the input
vertex.

    \begin{tcolorbox}[breakable, size=fbox, boxrule=1pt, pad at break*=1mm,colback=cellbackground, colframe=cellborder]
\prompt{In}{incolor}{ }{\boxspacing}
\begin{Verbatim}[commandchars=\\\{\}]
\PY{k}{def} \PY{n+nf}{vertexClusteringCoefficient}\PY{p}{(}\PY{n}{edges}\PY{p}{:} \PY{n+nb}{list}\PY{p}{,} \PY{n}{vertex}\PY{p}{:} \PY{n+nb}{int}\PY{p}{)}\PY{p}{:}
    \PY{n}{adjacent\PYZus{}vertices\PYZus{}list} \PY{o}{=} \PY{n}{vertexDegree}\PY{p}{(}\PY{n}{edges}\PY{o}{=}\PY{n}{edges}\PY{p}{,} \PY{n}{vertex}\PY{o}{=}\PY{n}{vertex}\PY{p}{,} \PY{n}{returnAdjacentVertexList}\PY{o}{=}\PY{k+kc}{True}\PY{p}{)}
    \PY{n}{neighbor\PYZus{}edge\PYZus{}dict} \PY{o}{=} \PY{n+nb}{dict}\PY{p}{(}\PY{p}{)}
    \PY{n}{edges\PYZus{}amoung\PYZus{}neighbors} \PY{o}{=} \PY{n+nb}{list}\PY{p}{(}\PY{p}{)}
    \PY{k}{for} \PY{n}{edge} \PY{o+ow}{in} \PY{n}{edges}\PY{p}{:}
        \PY{n}{p1} \PY{o}{=} \PY{n}{edge}\PY{p}{[}\PY{l+m+mi}{0}\PY{p}{]}
        \PY{n}{p2} \PY{o}{=} \PY{n}{edge}\PY{p}{[}\PY{l+m+mi}{1}\PY{p}{]}
        \PY{k}{if} \PY{n}{p1} \PY{o+ow}{not} \PY{o+ow}{in} \PY{n}{adjacent\PYZus{}vertices\PYZus{}list} \PY{o+ow}{or} \PY{n}{p2} \PY{o+ow}{not} \PY{o+ow}{in} \PY{n}{adjacent\PYZus{}vertices\PYZus{}list}\PY{p}{:} \PY{k}{continue}
        \PY{k}{if} \PY{n}{p2} \PY{o+ow}{in} \PY{n}{neighbor\PYZus{}edge\PYZus{}dict}\PY{p}{:}
            \PY{k}{if} \PY{n}{p1} \PY{o+ow}{not} \PY{o+ow}{in} \PY{n}{neighbor\PYZus{}edge\PYZus{}dict}\PY{p}{[}\PY{n}{p2}\PY{p}{]}\PY{p}{:}
                \PY{n}{neighbor\PYZus{}edge\PYZus{}dict}\PY{p}{[}\PY{n}{p2}\PY{p}{]}\PY{o}{.}\PY{n}{append}\PY{p}{(}\PY{n}{p1}\PY{p}{)}
            \PY{k}{continue}
        \PY{k}{if} \PY{n}{p1} \PY{o+ow}{not} \PY{o+ow}{in} \PY{n}{neighbor\PYZus{}edge\PYZus{}dict}\PY{p}{:}
            \PY{n}{neighbor\PYZus{}edge\PYZus{}dict}\PY{p}{[}\PY{n}{p1}\PY{p}{]} \PY{o}{=} \PY{n+nb}{list}\PY{p}{(}\PY{p}{)}
        \PY{k}{if} \PY{n}{p2} \PY{o+ow}{not} \PY{o+ow}{in} \PY{n}{neighbor\PYZus{}edge\PYZus{}dict}\PY{p}{[}\PY{n}{p1}\PY{p}{]}\PY{p}{:}
            \PY{n}{neighbor\PYZus{}edge\PYZus{}dict}\PY{p}{[}\PY{n}{p1}\PY{p}{]}\PY{o}{.}\PY{n}{append}\PY{p}{(}\PY{n}{p2}\PY{p}{)}
        \PY{n}{edges\PYZus{}amoung\PYZus{}neighbors}\PY{o}{.}\PY{n}{append}\PY{p}{(}\PY{n}{edge}\PY{p}{)}
    \PY{n}{num\PYZus{}edges\PYZus{}amoung\PYZus{}neighbors} \PY{o}{=} \PY{n+nb}{len}\PY{p}{(}\PY{n}{edges\PYZus{}amoung\PYZus{}neighbors}\PY{p}{)}
    \PY{n}{num\PYZus{}of\PYZus{}neighbor\PYZus{}vertices} \PY{o}{=} \PY{n+nb}{len}\PY{p}{(}\PY{n}{adjacent\PYZus{}vertices\PYZus{}list}\PY{p}{)}
    \PY{n}{answer} \PY{o}{=} \PY{l+m+mi}{0}
    \PY{c+c1}{\PYZsh{} avoid division by 0}
    \PY{k}{try}\PY{p}{:}
        \PY{n}{num\PYZus{}possible\PYZus{}edges} \PY{o}{=} \PY{p}{(}\PY{n}{num\PYZus{}of\PYZus{}neighbor\PYZus{}vertices} \PY{o}{*} \PY{p}{(}\PY{n}{num\PYZus{}of\PYZus{}neighbor\PYZus{}vertices} \PY{o}{\PYZhy{}} \PY{l+m+mi}{1}\PY{p}{)} \PY{p}{)} \PY{o}{/} \PY{l+m+mi}{2}
        \PY{n}{answer} \PY{o}{=} \PY{n}{num\PYZus{}edges\PYZus{}amoung\PYZus{}neighbors} \PY{o}{/} \PY{n}{num\PYZus{}possible\PYZus{}edges}
    \PY{k}{except}\PY{p}{:}
        \PY{k}{pass}
    \PY{k}{return} \PY{n}{answer}
\end{Verbatim}
\end{tcolorbox}

    \hypertarget{points-betweenness-centrality-of-a-vertex}{%
\subsection*{4. (5 points) Betweenness centrality of a
vertex}\label{points-betweenness-centrality-of-a-vertex}}

    A function that takes the following input: a list of edges representing
a graph, where each edge is a pair, and a vertex index that is an
integer. The output should be the betweenness centrality of the input
vertex.

    \begin{tcolorbox}[breakable, size=fbox, boxrule=1pt, pad at break*=1mm,colback=cellbackground, colframe=cellborder]
\prompt{In}{incolor}{ }{\boxspacing}
\begin{Verbatim}[commandchars=\\\{\}]
\PY{n}{shortest\PYZus{}path\PYZus{}dict} \PY{o}{=} \PY{n+nb}{dict}\PY{p}{(}\PY{p}{)}
\PY{n}{shortest\PYZus{}path\PYZus{}len\PYZus{}dict} \PY{o}{=} \PY{n+nb}{dict}\PY{p}{(}\PY{p}{)}
\PY{k}{def} \PY{n+nf}{getShortestPath}\PY{p}{(}\PY{n}{betweenness\PYZus{}G}\PY{p}{,} \PY{n}{v1}\PY{p}{,} \PY{n}{v2}\PY{p}{)}\PY{p}{:}
    \PY{k}{global} \PY{n}{shortest\PYZus{}path\PYZus{}dict}
    \PY{n}{lower\PYZus{}v} \PY{o}{=} \PY{n}{v1}
    \PY{n}{greater\PYZus{}v} \PY{o}{=} \PY{n}{v2}
    \PY{k}{if} \PY{n}{v1} \PY{o}{\PYZlt{}}\PY{o}{=} \PY{n}{v2}\PY{p}{:}
        \PY{n}{lower\PYZus{}v} \PY{o}{=} \PY{n}{v1}
        \PY{n}{greater\PYZus{}v} \PY{o}{=} \PY{n}{v2}
    \PY{k}{else}\PY{p}{:}
        \PY{n}{lower\PYZus{}v} \PY{o}{=} \PY{n}{v2}
        \PY{n}{greater\PYZus{}v} \PY{o}{=} \PY{n}{v1}

    \PY{k}{if} \PY{n}{lower\PYZus{}v} \PY{o+ow}{in} \PY{n}{shortest\PYZus{}path\PYZus{}dict} \PY{o+ow}{and} \PY{n}{greater\PYZus{}v} \PY{o+ow}{in} \PY{n}{shortest\PYZus{}path\PYZus{}dict}\PY{p}{[}\PY{n}{lower\PYZus{}v}\PY{p}{]}\PY{p}{:}
        \PY{k}{return} \PY{n}{shortest\PYZus{}path\PYZus{}dict}\PY{p}{[}\PY{n}{lower\PYZus{}v}\PY{p}{]}\PY{p}{[}\PY{n}{greater\PYZus{}v}\PY{p}{]}
    \PY{k}{elif} \PY{n}{greater\PYZus{}v} \PY{o+ow}{in} \PY{n}{shortest\PYZus{}path\PYZus{}dict} \PY{o+ow}{and} \PY{n}{lower\PYZus{}v} \PY{o+ow}{in} \PY{n}{shortest\PYZus{}path\PYZus{}dict}\PY{p}{[}\PY{n}{greater\PYZus{}v}\PY{p}{]}\PY{p}{:}
        \PY{k}{return} \PY{n}{shortest\PYZus{}path\PYZus{}dict}\PY{p}{[}\PY{n}{greater\PYZus{}v}\PY{p}{]}\PY{p}{[}\PY{n}{lower\PYZus{}v}\PY{p}{]}
    \PY{k}{else}\PY{p}{:}
        \PY{n}{shortest\PYZus{}path} \PY{o}{=} \PY{n}{nx}\PY{o}{.}\PY{n}{shortest\PYZus{}path}\PY{p}{(}\PY{n}{betweenness\PYZus{}G}\PY{p}{,} \PY{n}{lower\PYZus{}v}\PY{p}{,} \PY{n}{greater\PYZus{}v}\PY{p}{)}
        \PY{n}{shortest\PYZus{}path\PYZus{}dict}\PY{p}{[}\PY{n}{lower\PYZus{}v}\PY{p}{]}\PY{p}{[}\PY{n}{greater\PYZus{}v}\PY{p}{]} \PY{o}{=} \PY{n}{shortest\PYZus{}path}
        \PY{k}{return} \PY{n}{shortest\PYZus{}path}

\PY{k}{def} \PY{n+nf}{getShortestPathLength}\PY{p}{(}\PY{n}{betweenness\PYZus{}G}\PY{p}{,} \PY{n}{v1}\PY{p}{,} \PY{n}{v2}\PY{p}{)}\PY{p}{:}
    \PY{k}{global} \PY{n}{shortest\PYZus{}path\PYZus{}len\PYZus{}dict}
    \PY{n}{lower\PYZus{}v} \PY{o}{=} \PY{n}{v1}
    \PY{n}{greater\PYZus{}v} \PY{o}{=} \PY{n}{v2}
    \PY{k}{if} \PY{n}{v1} \PY{o}{\PYZlt{}}\PY{o}{=} \PY{n}{v2}\PY{p}{:}
        \PY{n}{lower\PYZus{}v} \PY{o}{=} \PY{n}{v1}
        \PY{n}{greater\PYZus{}v} \PY{o}{=} \PY{n}{v2}
    \PY{k}{else}\PY{p}{:}
        \PY{n}{lower\PYZus{}v} \PY{o}{=} \PY{n}{v2}
        \PY{n}{greater\PYZus{}v} \PY{o}{=} \PY{n}{v1}

    \PY{k}{if} \PY{n}{lower\PYZus{}v} \PY{o+ow}{in} \PY{n}{shortest\PYZus{}path\PYZus{}len\PYZus{}dict} \PY{o+ow}{and} \PY{n}{greater\PYZus{}v} \PY{o+ow}{in} \PY{n}{shortest\PYZus{}path\PYZus{}len\PYZus{}dict}\PY{p}{[}\PY{n}{lower\PYZus{}v}\PY{p}{]}\PY{p}{:}
        \PY{k}{return} \PY{n}{shortest\PYZus{}path\PYZus{}len\PYZus{}dict}\PY{p}{[}\PY{n}{lower\PYZus{}v}\PY{p}{]}\PY{p}{[}\PY{n}{greater\PYZus{}v}\PY{p}{]}
    \PY{k}{elif} \PY{n}{greater\PYZus{}v} \PY{o+ow}{in} \PY{n}{shortest\PYZus{}path\PYZus{}len\PYZus{}dict} \PY{o+ow}{and} \PY{n}{lower\PYZus{}v} \PY{o+ow}{in} \PY{n}{shortest\PYZus{}path\PYZus{}len\PYZus{}dict}\PY{p}{[}\PY{n}{greater\PYZus{}v}\PY{p}{]}\PY{p}{:}
        \PY{k}{return} \PY{n}{shortest\PYZus{}path\PYZus{}len\PYZus{}dict}\PY{p}{[}\PY{n}{greater\PYZus{}v}\PY{p}{]}\PY{p}{[}\PY{n}{lower\PYZus{}v}\PY{p}{]}
    \PY{k}{else}\PY{p}{:}
        \PY{n}{shortest\PYZus{}path\PYZus{}len} \PY{o}{=} \PY{n}{nx}\PY{o}{.}\PY{n}{shortest\PYZus{}path\PYZus{}length}\PY{p}{(}\PY{n}{betweenness\PYZus{}G}\PY{p}{,} \PY{n}{lower\PYZus{}v}\PY{p}{,} \PY{n}{greater\PYZus{}v}\PY{p}{)}
        \PY{n}{shortest\PYZus{}path\PYZus{}len\PYZus{}dict}\PY{p}{[}\PY{n}{lower\PYZus{}v}\PY{p}{]}\PY{p}{[}\PY{n}{greater\PYZus{}v}\PY{p}{]} \PY{o}{=} \PY{n}{shortest\PYZus{}path\PYZus{}len}
        \PY{k}{return} \PY{n}{shortest\PYZus{}path\PYZus{}len}

\PY{n}{betweenness\PYZus{}vertices} \PY{o}{=} \PY{k+kc}{None}
\PY{n}{betweenness\PYZus{}G} \PY{o}{=} \PY{k+kc}{None}

\PY{k}{def} \PY{n+nf}{vertexBetweennessCentrality}\PY{p}{(}\PY{n}{edges}\PY{p}{:} \PY{n+nb}{list}\PY{p}{,} \PY{n}{source\PYZus{}vertex}\PY{p}{:} \PY{n+nb}{int}\PY{p}{)}\PY{p}{:}
    \PY{k}{global} \PY{n}{betweenness\PYZus{}vertices}
    \PY{k}{global} \PY{n}{betweenness\PYZus{}G}
    \PY{k}{if} \PY{n}{betweenness\PYZus{}vertices} \PY{o+ow}{is} \PY{k+kc}{None}\PY{p}{:}
        \PY{n}{betweenness\PYZus{}vertices} \PY{o}{=} \PY{n+nb}{list}\PY{p}{(}\PY{p}{)}
        \PY{n}{betweenness\PYZus{}G} \PY{o}{=} \PY{n}{nx}\PY{o}{.}\PY{n}{Graph}\PY{p}{(}\PY{p}{)}
        \PY{k}{for} \PY{n}{edge} \PY{o+ow}{in} \PY{n}{edges}\PY{p}{:}
            \PY{n}{betweenness\PYZus{}vertices}\PY{o}{.}\PY{n}{append}\PY{p}{(}\PY{n}{edge}\PY{p}{[}\PY{l+m+mi}{0}\PY{p}{]}\PY{p}{)}
            \PY{n}{betweenness\PYZus{}vertices}\PY{o}{.}\PY{n}{append}\PY{p}{(}\PY{n}{edge}\PY{p}{[}\PY{l+m+mi}{1}\PY{p}{]}\PY{p}{)}
            \PY{n}{betweenness\PYZus{}G}\PY{o}{.}\PY{n}{add\PYZus{}edge}\PY{p}{(}\PY{n}{edge}\PY{p}{[}\PY{l+m+mi}{0}\PY{p}{]}\PY{p}{,} \PY{n}{edge}\PY{p}{[}\PY{l+m+mi}{1}\PY{p}{]}\PY{p}{)}
        \PY{n}{betweenness\PYZus{}vertices} \PY{o}{=} \PY{n+nb}{set}\PY{p}{(}\PY{n}{betweenness\PYZus{}vertices}\PY{p}{)}
    \PY{n}{vertices} \PY{o}{=} \PY{n}{betweenness\PYZus{}vertices}
    \PY{n}{already\PYZus{}compared} \PY{o}{=} \PY{n+nb}{dict}\PY{p}{(}\PY{p}{)}
    \PY{k}{def} \PY{n+nf}{hasBeenCompared}\PY{p}{(}\PY{n}{v1}\PY{p}{,} \PY{n}{v2}\PY{p}{)}\PY{p}{:}
        \PY{k}{if} \PY{n}{v1} \PY{o+ow}{not} \PY{o+ow}{in} \PY{n}{already\PYZus{}compared}\PY{p}{:} \PY{n}{already\PYZus{}compared}\PY{p}{[}\PY{n}{v1}\PY{p}{]} \PY{o}{=} \PY{n+nb}{list}\PY{p}{(}\PY{p}{)}
        \PY{k}{if} \PY{n}{v2} \PY{o+ow}{not} \PY{o+ow}{in} \PY{n}{already\PYZus{}compared}\PY{p}{[}\PY{n}{v1}\PY{p}{]}\PY{p}{:}
            \PY{n}{already\PYZus{}compared}\PY{p}{[}\PY{n}{v1}\PY{p}{]}\PY{o}{.}\PY{n}{append}\PY{p}{(}\PY{n}{v2}\PY{p}{)}
            \PY{k}{return} \PY{k+kc}{False}
        \PY{k}{if} \PY{n}{v2} \PY{o+ow}{in} \PY{n}{already\PYZus{}compared}\PY{p}{:}
            \PY{k}{if} \PY{n}{v1} \PY{o+ow}{not} \PY{o+ow}{in} \PY{n}{already\PYZus{}compared}\PY{p}{[}\PY{n}{v2}\PY{p}{]}\PY{p}{:}
                \PY{n}{already\PYZus{}compared}\PY{p}{[}\PY{n}{v2}\PY{p}{]}\PY{o}{.}\PY{n}{append}\PY{p}{(}\PY{n}{v1}\PY{p}{)}
                \PY{k}{return} \PY{k+kc}{False}
        \PY{k}{return} \PY{k+kc}{True}

    \PY{n}{answer} \PY{o}{=} \PY{l+m+mi}{0}
    \PY{k}{for} \PY{n}{vertex1} \PY{o+ow}{in} \PY{n}{vertices}\PY{p}{:}
        \PY{k}{if} \PY{n}{vertex1} \PY{o}{==} \PY{n}{source\PYZus{}vertex}\PY{p}{:} \PY{k}{continue}
        \PY{k}{for} \PY{n}{vertex2} \PY{o+ow}{in} \PY{n}{vertices}\PY{p}{:}
            \PY{k}{if} \PY{n}{vertex2} \PY{o}{==} \PY{n}{source\PYZus{}vertex}\PY{p}{:} \PY{k}{continue}
            \PY{k}{if} \PY{n}{hasBeenCompared}\PY{p}{(}\PY{n}{vertex1}\PY{p}{,} \PY{n}{vertex2}\PY{p}{)}\PY{p}{:} \PY{k}{continue}
            \PY{n}{sub\PYZus{}top} \PY{o}{=} \PY{l+m+mi}{0}
            \PY{n}{sub\PYZus{}bottom} \PY{o}{=} \PY{l+m+mi}{0}
            \PY{n}{sub\PYZus{}answer} \PY{o}{=} \PY{l+m+mi}{0}
            \PY{k}{try}\PY{p}{:}
                \PY{n}{shortest\PYZus{}path} \PY{o}{=} \PY{n}{getShortestPath}\PY{p}{(}\PY{n}{betweenness\PYZus{}G}\PY{p}{,} \PY{n}{vertex1}\PY{p}{,} \PY{n}{vertex2}\PY{p}{)}
                \PY{n}{shortest\PYZus{}path\PYZus{}len} \PY{o}{=} \PY{n}{getShortestPathLength}\PY{p}{(}\PY{n}{betweenness\PYZus{}G}\PY{p}{,} \PY{n}{vertex1}\PY{p}{,} \PY{n}{vertex2}\PY{p}{)}
                \PY{c+c1}{\PYZsh{} shortest\PYZus{}path = nx.shortest\PYZus{}path(G, vertex1, vertex2)}
                \PY{c+c1}{\PYZsh{} shortest\PYZus{}path\PYZus{}len = nx.shortest\PYZus{}path\PYZus{}length(G, vertex1, vertex2)}
                \PY{k}{if} \PY{n}{source\PYZus{}vertex} \PY{o+ow}{in} \PY{n}{shortest\PYZus{}path}\PY{p}{:}
                    \PY{n}{sub\PYZus{}top} \PY{o}{+}\PY{o}{=} \PY{l+m+mi}{1}
                \PY{k}{if} \PY{n}{shortest\PYZus{}path\PYZus{}len} \PY{o}{\PYZgt{}} \PY{l+m+mi}{0}\PY{p}{:}
                    \PY{n}{sub\PYZus{}bottom} \PY{o}{+}\PY{o}{=} \PY{l+m+mi}{1}
                \PY{n}{sub\PYZus{}answer} \PY{o}{=} \PY{n}{sub\PYZus{}top} \PY{o}{/} \PY{n}{sub\PYZus{}bottom}
            \PY{k}{except}\PY{p}{:}
                \PY{k}{continue}
            \PY{n}{answer} \PY{o}{+}\PY{o}{=} \PY{n}{sub\PYZus{}answer}

    \PY{k}{return} \PY{n}{answer}
\end{Verbatim}
\end{tcolorbox}

    \hypertarget{points-adjacency-matrix}{%
\subsection*{5. (5 points) Adjacency
matrix}\label{points-adjacency-matrix}}

    A function that takes the following input: a list of edges representing
a graph, where each edge is a pair. The output should be the dense
adjacency matrix of the graph.

    \begin{tcolorbox}[breakable, size=fbox, boxrule=1pt, pad at break*=1mm,colback=cellbackground, colframe=cellborder]
\prompt{In}{incolor}{ }{\boxspacing}
\begin{Verbatim}[commandchars=\\\{\}]
\PY{k}{def} \PY{n+nf}{adjacencyMatrix}\PY{p}{(}\PY{n}{edges}\PY{p}{:} \PY{n+nb}{list}\PY{p}{)}\PY{p}{:}
    \PY{n}{num\PYZus{}of\PYZus{}vertices} \PY{o}{=} \PY{n}{numberOfVertices}\PY{p}{(}\PY{n}{edges}\PY{o}{=}\PY{n}{edges}\PY{p}{)}
    \PY{n}{m} \PY{o}{=} \PY{n}{np}\PY{o}{.}\PY{n}{ndarray}\PY{p}{(}\PY{n}{shape}\PY{o}{=}\PY{p}{(}\PY{n}{num\PYZus{}of\PYZus{}vertices}\PY{p}{,} \PY{n}{num\PYZus{}of\PYZus{}vertices}\PY{p}{)}\PY{p}{,} \PY{n}{dtype}\PY{o}{=}\PY{n+nb}{int}\PY{p}{)} \PY{c+c1}{\PYZsh{} output matrix}
    \PY{c+c1}{\PYZsh{} zero values}
    \PY{k}{for} \PY{n}{row} \PY{o+ow}{in} \PY{n+nb}{range}\PY{p}{(}\PY{n}{m}\PY{o}{.}\PY{n}{shape}\PY{p}{[}\PY{l+m+mi}{0}\PY{p}{]}\PY{p}{)}\PY{p}{:}
        \PY{n}{m}\PY{p}{[}\PY{n}{row}\PY{p}{]} \PY{o}{=} \PY{n}{np}\PY{o}{.}\PY{n}{array}\PY{p}{(}\PY{p}{[}\PY{l+m+mi}{0}\PY{p}{]} \PY{o}{*} \PY{n}{m}\PY{o}{.}\PY{n}{shape}\PY{p}{[}\PY{l+m+mi}{1}\PY{p}{]}\PY{p}{)}
    \PY{k}{for} \PY{n}{edge} \PY{o+ow}{in} \PY{n}{edges}\PY{p}{:}
        \PY{n}{m}\PY{p}{[}\PY{n}{edge}\PY{p}{[}\PY{l+m+mi}{0}\PY{p}{]} \PY{o}{\PYZhy{}} \PY{l+m+mi}{1}\PY{p}{]}\PY{p}{[}\PY{n}{edge}\PY{p}{[}\PY{l+m+mi}{1}\PY{p}{]} \PY{o}{\PYZhy{}} \PY{l+m+mi}{1}\PY{p}{]} \PY{o}{=} \PY{l+m+mi}{1}
    \PY{k}{return} \PY{n}{m}
\end{Verbatim}
\end{tcolorbox}

    \hypertarget{points-prestige-centrality-of-vertices}{%
\subsection*{6. (10 points) Prestige centrality of
vertices}\label{points-prestige-centrality-of-vertices}}

    A function that takes the following input: a dense adjacency matrix
representation of a graph. The output should be the prestige values for
each vertex in the graph. (Note, you may NOT use linear algebra
functions in numpy, spicy, or any other library to find eigenvectors but
you MAY use linear algebra functions for transpose, matrix-vector
multiplication, computing the dot product, the norm, and argmax.)

    \begin{tcolorbox}[breakable, size=fbox, boxrule=1pt, pad at break*=1mm,colback=cellbackground, colframe=cellborder]
\prompt{In}{incolor}{ }{\boxspacing}
\begin{Verbatim}[commandchars=\\\{\}]
\PY{k}{def} \PY{n+nf}{prestige\PYZus{}centrality}\PY{p}{(}\PY{n}{m}\PY{p}{:} \PY{n}{np}\PY{o}{.}\PY{n}{matrix}\PY{p}{,} \PY{n}{max\PYZus{}iter} \PY{o}{=} \PY{l+m+mi}{100000}\PY{p}{)}\PY{p}{:}
    \PY{c+c1}{\PYZsh{} validate matrix data type}
    \PY{k}{if} \PY{o+ow}{not} \PY{n+nb}{isinstance}\PY{p}{(}\PY{n}{m}\PY{p}{,} \PY{n}{np}\PY{o}{.}\PY{n}{matrix}\PY{p}{)}\PY{p}{:}
        \PY{k}{raise} \PY{n+ne}{TypeError}\PY{p}{(}\PY{l+s+s2}{\PYZdq{}}\PY{l+s+s2}{matrix `m` must be type `np.matrix`}\PY{l+s+s2}{\PYZdq{}}\PY{p}{)}
    \PY{c+c1}{\PYZsh{} transpose matrix}
    \PY{n}{a} \PY{o}{=} \PY{n}{m}\PY{o}{.}\PY{n}{transpose}\PY{p}{(}\PY{p}{)}
    \PY{c+c1}{\PYZsh{} a = m}

    \PY{n}{p\PYZus{}0} \PY{o}{=} \PY{n}{np}\PY{o}{.}\PY{n}{array}\PY{p}{(}\PY{p}{[}\PY{l+m+mi}{1}\PY{p}{]} \PY{o}{*} \PY{n}{a}\PY{o}{.}\PY{n}{shape}\PY{p}{[}\PY{l+m+mi}{0}\PY{p}{]}\PY{p}{)}
    \PY{n}{p\PYZus{}arr} \PY{o}{=} \PY{n+nb}{list}\PY{p}{(}\PY{p}{)}
    \PY{n}{p\PYZus{}arr}\PY{o}{.}\PY{n}{append}\PY{p}{(}\PY{n}{p\PYZus{}0}\PY{p}{)}


    \PY{c+c1}{\PYZsh{} power iteration}
    \PY{n}{iter\PYZus{}count} \PY{o}{=} \PY{l+m+mi}{0}
    \PY{k}{while} \PY{k+kc}{True}\PY{p}{:}
        \PY{n}{iter\PYZus{}count} \PY{o}{+}\PY{o}{=} \PY{l+m+mi}{1}
        \PY{k}{if} \PY{n}{iter\PYZus{}count} \PY{o}{\PYZgt{}} \PY{n}{max\PYZus{}iter}\PY{p}{:} \PY{k}{break}
        \PY{n}{p\PYZus{}j} \PY{o}{=} \PY{p}{[}\PY{l+m+mi}{0}\PY{p}{]} \PY{o}{*} \PY{n}{a}\PY{o}{.}\PY{n}{shape}\PY{p}{[}\PY{l+m+mi}{0}\PY{p}{]}
        \PY{n}{p\PYZus{}i} \PY{o}{=} \PY{n}{p\PYZus{}arr}\PY{p}{[}\PY{n+nb}{len}\PY{p}{(}\PY{n}{p\PYZus{}arr}\PY{p}{)} \PY{o}{\PYZhy{}} \PY{l+m+mi}{1}\PY{p}{]}
        \PY{n}{j\PYZus{}index} \PY{o}{=} \PY{l+m+mi}{0}
        \PY{k}{for} \PY{n}{i} \PY{o+ow}{in} \PY{n}{a}\PY{p}{:}
            \PY{n}{a\PYZus{}i\PYZus{}row} \PY{o}{=} \PY{n}{np}\PY{o}{.}\PY{n}{array}\PY{p}{(}\PY{n}{i}\PY{p}{)}\PY{o}{.}\PY{n}{flatten}\PY{p}{(}\PY{p}{)}
            \PY{n}{p\PYZus{}j}\PY{p}{[}\PY{n}{j\PYZus{}index}\PY{p}{]} \PY{o}{=} \PY{n}{np}\PY{o}{.}\PY{n}{dot}\PY{p}{(}\PY{n}{a\PYZus{}i\PYZus{}row}\PY{p}{,} \PY{n}{p\PYZus{}i}\PY{p}{)}
            \PY{n}{j\PYZus{}index} \PY{o}{+}\PY{o}{=} \PY{l+m+mi}{1}
        \PY{c+c1}{\PYZsh{} normalize the vector}
        \PY{n}{p\PYZus{}j} \PY{o}{=} \PY{n}{p\PYZus{}j} \PY{o}{/} \PY{n}{np}\PY{o}{.}\PY{n}{amax}\PY{p}{(}\PY{n}{p\PYZus{}j}\PY{p}{)}
        \PY{n}{p\PYZus{}arr}\PY{o}{.}\PY{n}{append}\PY{p}{(}\PY{n}{p\PYZus{}j}\PY{p}{)}
        \PY{n}{converged} \PY{o}{=} \PY{k+kc}{True}
        \PY{k}{for} \PY{n}{p\PYZus{}j\PYZus{}index} \PY{o+ow}{in} \PY{n+nb}{range}\PY{p}{(}\PY{n+nb}{len}\PY{p}{(}\PY{n}{p\PYZus{}j}\PY{p}{)}\PY{p}{)}\PY{p}{:}
            \PY{n}{j\PYZus{}val} \PY{o}{=} \PY{n}{p\PYZus{}j}\PY{p}{[}\PY{n}{p\PYZus{}j\PYZus{}index}\PY{p}{]}
            \PY{n}{i\PYZus{}val} \PY{o}{=} \PY{n}{p\PYZus{}i}\PY{p}{[}\PY{n}{p\PYZus{}j\PYZus{}index}\PY{p}{]}
            \PY{n}{i\PYZus{}j\PYZus{}diff} \PY{o}{=} \PY{n+nb}{abs}\PY{p}{(}\PY{n}{j\PYZus{}val} \PY{o}{\PYZhy{}} \PY{n}{i\PYZus{}val}\PY{p}{)}
            \PY{k}{if} \PY{n}{i\PYZus{}j\PYZus{}diff} \PY{o}{\PYZgt{}} \PY{l+m+mf}{0.00001}\PY{p}{:}
                \PY{n}{converged} \PY{o}{=} \PY{k+kc}{False}
                \PY{k}{break}
        \PY{k}{if} \PY{n}{converged} \PY{o+ow}{is} \PY{k+kc}{True}\PY{p}{:}
            \PY{k}{break}

    \PY{k}{return} \PY{n}{p\PYZus{}arr}\PY{p}{[}\PY{n+nb}{len}\PY{p}{(}\PY{n}{p\PYZus{}arr}\PY{p}{)} \PY{o}{\PYZhy{}} \PY{l+m+mi}{1}\PY{p}{]}
\end{Verbatim}
\end{tcolorbox}

    \hypertarget{part-3-analyze-the-graph-data}{%
\section*{Part 3: Analyze the graph
data}\label{part-3-analyze-the-graph-data}}

    Using tables or figures as appropriate, report the following. You may
treat the graph as a simple undirected, unweighted graph. You may use
networkx functions in all of Part 3, but you are encouraged to test out
your functions from Part 2 on real-world data.

    \hypertarget{code-setup}{%
\subsection*{Code Setup}\label{code-setup}}

    Get list of edges from the real world data.

    \begin{tcolorbox}[breakable, size=fbox, boxrule=1pt, pad at break*=1mm,colback=cellbackground, colframe=cellborder]
\prompt{In}{incolor}{ }{\boxspacing}
\begin{Verbatim}[commandchars=\\\{\}]
\PY{n}{edges} \PY{o}{=} \PY{n}{getFileData}\PY{p}{(}\PY{n}{DATA\PYZus{}URL}\PY{p}{)}
\end{Verbatim}
\end{tcolorbox}

    Create a NetworkX Graph from the real world data.

    \begin{tcolorbox}[breakable, size=fbox, boxrule=1pt, pad at break*=1mm,colback=cellbackground, colframe=cellborder]
\prompt{In}{incolor}{ }{\boxspacing}
\begin{Verbatim}[commandchars=\\\{\}]
\PY{c+c1}{\PYZsh{} G will be the complete graph of all the edged from the real world data.}
\PY{n}{G} \PY{o}{=} \PY{n}{nx}\PY{o}{.}\PY{n}{Graph}\PY{p}{(}\PY{p}{)}
\PY{c+c1}{\PYZsh{} add edges to the graph}
\PY{k}{for} \PY{n}{edge} \PY{o+ow}{in} \PY{n}{edges}\PY{p}{:} \PY{n}{G}\PY{o}{.}\PY{n}{add\PYZus{}edge}\PY{p}{(}\PY{n}{edge}\PY{p}{[}\PY{l+m+mi}{0}\PY{p}{]}\PY{p}{,} \PY{n}{edge}\PY{p}{[}\PY{l+m+mi}{1}\PY{p}{]}\PY{p}{)}
\PY{c+c1}{\PYZsh{} Show graph info}
\PY{n}{nx}\PY{o}{.}\PY{n}{info}\PY{p}{(}\PY{n}{G}\PY{p}{)}
\end{Verbatim}
\end{tcolorbox}

            \begin{tcolorbox}[breakable, size=fbox, boxrule=.5pt, pad at break*=1mm, opacityfill=0]
\prompt{Out}{outcolor}{ }{\boxspacing}
\begin{Verbatim}[commandchars=\\\{\}]
'Graph with 4039 nodes and 88234 edges'
\end{Verbatim}
\end{tcolorbox}
        
    \hypertarget{helper-functions}{%
\subsubsection*{Helper Functions}\label{helper-functions}}

    The following are some common methods that will be used.

    \begin{tcolorbox}[breakable, size=fbox, boxrule=1pt, pad at break*=1mm,colback=cellbackground, colframe=cellborder]
\prompt{In}{incolor}{ }{\boxspacing}
\begin{Verbatim}[commandchars=\\\{\}]
\PY{c+c1}{\PYZsh{} function to sort a dictionary by its values}
\PY{k}{def} \PY{n+nf}{sortDictionaryByValues}\PY{p}{(}\PY{n}{d}\PY{p}{:} \PY{n+nb}{dict}\PY{p}{,} \PY{n}{ascending}\PY{p}{:} \PY{n+nb}{bool} \PY{o}{=} \PY{k+kc}{True}\PY{p}{)}\PY{p}{:}
    \PY{c+c1}{\PYZsh{} validate the input `d` is type dictionary}
    \PY{k}{if} \PY{o+ow}{not} \PY{n+nb}{isinstance}\PY{p}{(}\PY{n}{d}\PY{p}{,} \PY{n+nb}{dict}\PY{p}{)}\PY{p}{:}
        \PY{k}{raise} \PY{n+ne}{TypeError}\PY{p}{(}\PY{l+s+s2}{\PYZdq{}}\PY{l+s+s2}{`d` must be type `dict`.}\PY{l+s+s2}{\PYZdq{}}\PY{p}{)}
    \PY{c+c1}{\PYZsh{} return the sorted dictionary}
    \PY{k}{return} \PY{p}{\PYZob{}}\PY{n}{k}\PY{p}{:} \PY{n}{v} \PY{k}{for} \PY{n}{k}\PY{p}{,} \PY{n}{v} \PY{o+ow}{in} \PY{n+nb}{sorted}\PY{p}{(}\PY{n}{d}\PY{o}{.}\PY{n}{items}\PY{p}{(}\PY{p}{)}\PY{p}{,} \PY{n}{key}\PY{o}{=}\PY{k}{lambda} \PY{n}{item}\PY{p}{:} \PY{n}{item}\PY{p}{[}\PY{l+m+mi}{1}\PY{p}{]}\PY{p}{,} \PY{n}{reverse}\PY{o}{=}\PY{n}{ascending}\PY{p}{)}\PY{p}{\PYZcb{}}
\end{Verbatim}
\end{tcolorbox}

    \begin{tcolorbox}[breakable, size=fbox, boxrule=1pt, pad at break*=1mm,colback=cellbackground, colframe=cellborder]
\prompt{In}{incolor}{ }{\boxspacing}
\begin{Verbatim}[commandchars=\\\{\}]
\PY{c+c1}{\PYZsh{} this method returns a dictionary containing the first elements}
\PY{c+c1}{\PYZsh{} through a given limit. (i.e. elements 0\PYZhy{}n)}
\PY{k}{def} \PY{n+nf}{limitNumDictionaryElements}\PY{p}{(}\PY{n}{d}\PY{p}{:} \PY{n+nb}{dict}\PY{p}{,} \PY{n}{limit}\PY{p}{:} \PY{n+nb}{int} \PY{o}{=} \PY{l+m+mi}{10}\PY{p}{)}\PY{p}{:}
    \PY{c+c1}{\PYZsh{} validate the input `d` is type dictionary}
    \PY{k}{if} \PY{o+ow}{not} \PY{n+nb}{isinstance}\PY{p}{(}\PY{n}{d}\PY{p}{,} \PY{n+nb}{dict}\PY{p}{)}\PY{p}{:}
        \PY{k}{raise} \PY{n+ne}{TypeError}\PY{p}{(}\PY{l+s+s2}{\PYZdq{}}\PY{l+s+s2}{`d` must be type `dict`.}\PY{l+s+s2}{\PYZdq{}}\PY{p}{)}
    \PY{c+c1}{\PYZsh{} return the limited dictionary}
    \PY{k}{return} \PY{p}{\PYZob{}}\PY{n}{k}\PY{p}{:} \PY{n}{v} \PY{k}{for} \PY{n}{k}\PY{p}{,} \PY{n}{v} \PY{o+ow}{in} \PY{n+nb}{list}\PY{p}{(}\PY{n}{d}\PY{o}{.}\PY{n}{items}\PY{p}{(}\PY{p}{)}\PY{p}{)}\PY{p}{[}\PY{p}{:}\PY{n}{limit}\PY{p}{]}\PY{p}{\PYZcb{}}
\end{Verbatim}
\end{tcolorbox}

    \begin{tcolorbox}[breakable, size=fbox, boxrule=1pt, pad at break*=1mm,colback=cellbackground, colframe=cellborder]
\prompt{In}{incolor}{ }{\boxspacing}
\begin{Verbatim}[commandchars=\\\{\}]
\PY{c+c1}{\PYZsh{} This method combines sortDictionaryByValues() and limitNumDictionaryElements()}
\PY{k}{def} \PY{n+nf}{topSortedDict}\PY{p}{(}\PY{n}{d}\PY{p}{:} \PY{n+nb}{dict}\PY{p}{,} \PY{n}{ascending}\PY{p}{:} \PY{n+nb}{bool} \PY{o}{=} \PY{k+kc}{True}\PY{p}{,} \PY{n}{limit}\PY{p}{:} \PY{n+nb}{int} \PY{o}{=} \PY{l+m+mi}{10}\PY{p}{)}\PY{p}{:}
    \PY{c+c1}{\PYZsh{} validate the input `d` is type dictionary}
    \PY{k}{if} \PY{o+ow}{not} \PY{n+nb}{isinstance}\PY{p}{(}\PY{n}{d}\PY{p}{,} \PY{n+nb}{dict}\PY{p}{)}\PY{p}{:}
        \PY{k}{raise} \PY{n+ne}{TypeError}\PY{p}{(}\PY{l+s+s2}{\PYZdq{}}\PY{l+s+s2}{`d` must be type `dict`.}\PY{l+s+s2}{\PYZdq{}}\PY{p}{)}
    \PY{k}{return} \PY{n}{limitNumDictionaryElements}\PY{p}{(}\PY{n}{sortDictionaryByValues}\PY{p}{(}\PY{n}{d}\PY{p}{,} \PY{n}{ascending}\PY{p}{)}\PY{p}{,} \PY{n}{limit}\PY{p}{)}
\end{Verbatim}
\end{tcolorbox}

    \begin{tcolorbox}[breakable, size=fbox, boxrule=1pt, pad at break*=1mm,colback=cellbackground, colframe=cellborder]
\prompt{In}{incolor}{ }{\boxspacing}
\begin{Verbatim}[commandchars=\\\{\}]
\PY{c+c1}{\PYZsh{} converts a dictionary to a sorted list where each}
\PY{c+c1}{\PYZsh{} elements of the list is a pair of the key\PYZhy{}values}
\PY{c+c1}{\PYZsh{} from the provided dictionary}
\PY{k}{def} \PY{n+nf}{convertDictToSortedList}\PY{p}{(}\PY{n}{d}\PY{p}{:} \PY{n+nb}{dict}\PY{p}{)}\PY{p}{:}
    \PY{k}{return} \PY{n+nb}{sorted}\PY{p}{(}\PY{n}{d}\PY{o}{.}\PY{n}{items}\PY{p}{(}\PY{p}{)}\PY{p}{,} \PY{n}{key}\PY{o}{=}\PY{k}{lambda} \PY{n}{item}\PY{p}{:} \PY{n}{item}\PY{p}{[}\PY{l+m+mi}{1}\PY{p}{]}\PY{p}{,} \PY{n}{reverse}\PY{o}{=}\PY{k+kc}{True}\PY{p}{)}
\end{Verbatim}
\end{tcolorbox}

    \hypertarget{points-graph-visualization}{%
\subsection*{1. (5 points) Graph
Visualization}\label{points-graph-visualization}}

    Produce a visualization of the graph (or graph sample that you used).

    \begin{tcolorbox}[breakable, size=fbox, boxrule=1pt, pad at break*=1mm,colback=cellbackground, colframe=cellborder]
\prompt{In}{incolor}{ }{\boxspacing}
\begin{Verbatim}[commandchars=\\\{\}]
\PY{n}{nx}\PY{o}{.}\PY{n}{draw\PYZus{}networkx}\PY{p}{(}\PY{n}{G}\PY{p}{,} \PY{n}{with\PYZus{}labels}\PY{o}{=}\PY{k+kc}{False}\PY{p}{,} \PY{n}{node\PYZus{}size}\PY{o}{=}\PY{l+m+mi}{5}\PY{p}{,} \PY{n}{width}\PY{o}{=}\PY{l+m+mi}{1}\PY{p}{)}
\end{Verbatim}
\end{tcolorbox}

    \begin{center}
    \adjustimage{max size={0.9\linewidth}{0.9\paperheight}}{project-02_files/project-02_43_0.png}
    \end{center}
    { \hspace*{\fill} \\}
    
    \hypertarget{points-top-10-highest-degree-nodes}{%
\subsection*{2. (3 points) Top 10 Highest Degree
Nodes}\label{points-top-10-highest-degree-nodes}}

    Find the 10 nodes with the highest degree.

    \begin{tcolorbox}[breakable, size=fbox, boxrule=1pt, pad at break*=1mm,colback=cellbackground, colframe=cellborder]
\prompt{In}{incolor}{ }{\boxspacing}
\begin{Verbatim}[commandchars=\\\{\}]
\PY{n}{verticies\PYZus{}degrees\PYZus{}dict} \PY{o}{=} \PY{n+nb}{dict}\PY{p}{(}\PY{p}{)}
\PY{c+c1}{\PYZsh{} calculate the degree for each vertex}
\PY{k}{for} \PY{n}{v} \PY{o+ow}{in} \PY{n+nb}{list}\PY{p}{(}\PY{n}{G}\PY{o}{.}\PY{n}{nodes}\PY{p}{)}\PY{p}{:}
    \PY{n}{verticies\PYZus{}degrees\PYZus{}dict}\PY{p}{[}\PY{n}{v}\PY{p}{]} \PY{o}{=} \PY{n}{vertexDegree}\PY{p}{(}\PY{n}{edges}\PY{p}{,} \PY{n}{v}\PY{p}{)}
\PY{c+c1}{\PYZsh{} get the top 10 highest values}
\PY{n}{top\PYZus{}10\PYZus{}highest\PYZus{}degree} \PY{o}{=} \PY{n}{topSortedDict}\PY{p}{(}\PY{n}{verticies\PYZus{}degrees\PYZus{}dict}\PY{p}{)}
\end{Verbatim}
\end{tcolorbox}

    \begin{tcolorbox}[breakable, size=fbox, boxrule=1pt, pad at break*=1mm,colback=cellbackground, colframe=cellborder]
\prompt{In}{incolor}{ }{\boxspacing}
\begin{Verbatim}[commandchars=\\\{\}]
\PY{c+c1}{\PYZsh{} 10 nodes with the highest degree}
\PY{n+nb}{list}\PY{p}{(}\PY{n}{top\PYZus{}10\PYZus{}highest\PYZus{}degree}\PY{o}{.}\PY{n}{keys}\PY{p}{(}\PY{p}{)}\PY{p}{)}
\end{Verbatim}
\end{tcolorbox}

            \begin{tcolorbox}[breakable, size=fbox, boxrule=.5pt, pad at break*=1mm, opacityfill=0]
\prompt{Out}{outcolor}{ }{\boxspacing}
\begin{Verbatim}[commandchars=\\\{\}]
[107, 1684, 1912, 3437, 0, 2543, 2347, 1888, 1800, 1663]
\end{Verbatim}
\end{tcolorbox}
        
    \begin{tcolorbox}[breakable, size=fbox, boxrule=1pt, pad at break*=1mm,colback=cellbackground, colframe=cellborder]
\prompt{In}{incolor}{ }{\boxspacing}
\begin{Verbatim}[commandchars=\\\{\}]
\PY{c+c1}{\PYZsh{} top 10 nodes with the highest degree \PYZhy{} (node, degree)}
\PY{n}{convertDictToSortedList}\PY{p}{(}\PY{n}{top\PYZus{}10\PYZus{}highest\PYZus{}degree}\PY{p}{)}
\end{Verbatim}
\end{tcolorbox}

            \begin{tcolorbox}[breakable, size=fbox, boxrule=.5pt, pad at break*=1mm, opacityfill=0]
\prompt{Out}{outcolor}{ }{\boxspacing}
\begin{Verbatim}[commandchars=\\\{\}]
[(107, 1045),
 (1684, 792),
 (1912, 755),
 (3437, 547),
 (0, 347),
 (2543, 294),
 (2347, 291),
 (1888, 254),
 (1800, 245),
 (1663, 235)]
\end{Verbatim}
\end{tcolorbox}
        
    \hypertarget{points-top-10-highest-betweenness-centrality}{%
\subsection*{3. (3 points) Top 10 Highest Betweenness
Centrality}\label{points-top-10-highest-betweenness-centrality}}

    Find the 10 nodes with the highest betweenness centrality.

    \begin{tcolorbox}[breakable, size=fbox, boxrule=1pt, pad at break*=1mm,colback=cellbackground, colframe=cellborder]
\prompt{In}{incolor}{ }{\boxspacing}
\begin{Verbatim}[commandchars=\\\{\}]
\PY{n}{topSortedDict}\PY{p}{(}\PY{n}{nx}\PY{o}{.}\PY{n}{betweenness\PYZus{}centrality}\PY{p}{(}\PY{n}{G}\PY{p}{,} \PY{n}{normalized}\PY{o}{=}\PY{k+kc}{False}\PY{p}{)}\PY{p}{)}
\end{Verbatim}
\end{tcolorbox}

            \begin{tcolorbox}[breakable, size=fbox, boxrule=.5pt, pad at break*=1mm, opacityfill=0]
\prompt{Out}{outcolor}{ }{\boxspacing}
\begin{Verbatim}[commandchars=\\\{\}]
\{0: 1192496.1130793944,
 58: 687594.983374667,
 107: 3916560.144440749,
 428: 524164.06777575763,
 567: 784996.9055941283,
 698: 940024.2464822,
 1085: 1214577.7583604807,
 1684: 2753286.686908284,
 1912: 1868918.212256787,
 3437: 1924506.1515714952\}
\end{Verbatim}
\end{tcolorbox}
        
    \hypertarget{points-top-10-highest-clustering-coefficient}{%
\subsection*{4. (3 points) Top 10 Highest Clustering
Coefficient}\label{points-top-10-highest-clustering-coefficient}}

    Find the 10 nodes with the highest clustering coefficient. If there are
ties, choose 10 to report and explain how the 10 were chosen.

    \begin{tcolorbox}[breakable, size=fbox, boxrule=1pt, pad at break*=1mm,colback=cellbackground, colframe=cellborder]
\prompt{In}{incolor}{ }{\boxspacing}
\begin{Verbatim}[commandchars=\\\{\}]
\PY{n}{verticies\PYZus{}clustering\PYZus{}coefficient\PYZus{}dict} \PY{o}{=} \PY{n+nb}{dict}\PY{p}{(}\PY{p}{)}
\PY{c+c1}{\PYZsh{} calculate the clustering coefficient for each vertex}
\PY{k}{for} \PY{n}{v} \PY{o+ow}{in} \PY{n+nb}{list}\PY{p}{(}\PY{n}{G}\PY{o}{.}\PY{n}{nodes}\PY{p}{)}\PY{p}{:}
    \PY{n}{verticies\PYZus{}clustering\PYZus{}coefficient\PYZus{}dict}\PY{p}{[}\PY{n}{v}\PY{p}{]} \PY{o}{=} \PY{n}{vertexClusteringCoefficient}\PY{p}{(}\PY{n}{edges}\PY{p}{,} \PY{n}{v}\PY{p}{)}
\PY{c+c1}{\PYZsh{} get the top 10 highest values}
\PY{n}{top\PYZus{}10\PYZus{}highest\PYZus{}clustering\PYZus{}coefficient} \PY{o}{=} \PY{n}{topSortedDict}\PY{p}{(}\PY{n}{verticies\PYZus{}clustering\PYZus{}coefficient\PYZus{}dict}\PY{p}{)}
\end{Verbatim}
\end{tcolorbox}

    \begin{tcolorbox}[breakable, size=fbox, boxrule=1pt, pad at break*=1mm,colback=cellbackground, colframe=cellborder]
\prompt{In}{incolor}{ }{\boxspacing}
\begin{Verbatim}[commandchars=\\\{\}]
\PY{c+c1}{\PYZsh{} 10 nodes with the greatest clustering coefficient}
\PY{n+nb}{list}\PY{p}{(}\PY{n}{top\PYZus{}10\PYZus{}highest\PYZus{}clustering\PYZus{}coefficient}\PY{o}{.}\PY{n}{keys}\PY{p}{(}\PY{p}{)}\PY{p}{)}
\end{Verbatim}
\end{tcolorbox}

            \begin{tcolorbox}[breakable, size=fbox, boxrule=.5pt, pad at break*=1mm, opacityfill=0]
\prompt{Out}{outcolor}{ }{\boxspacing}
\begin{Verbatim}[commandchars=\\\{\}]
[32, 33, 35, 42, 44, 46, 47, 52, 63, 70]
\end{Verbatim}
\end{tcolorbox}
        
    \begin{tcolorbox}[breakable, size=fbox, boxrule=1pt, pad at break*=1mm,colback=cellbackground, colframe=cellborder]
\prompt{In}{incolor}{ }{\boxspacing}
\begin{Verbatim}[commandchars=\\\{\}]
\PY{c+c1}{\PYZsh{} top 10 nodes with the clustering coefficient \PYZhy{} (node, degree)}
\PY{n}{convertDictToSortedList}\PY{p}{(}\PY{n}{top\PYZus{}10\PYZus{}highest\PYZus{}clustering\PYZus{}coefficient}\PY{p}{)}
\end{Verbatim}
\end{tcolorbox}

            \begin{tcolorbox}[breakable, size=fbox, boxrule=.5pt, pad at break*=1mm, opacityfill=0]
\prompt{Out}{outcolor}{ }{\boxspacing}
\begin{Verbatim}[commandchars=\\\{\}]
[(32, 1.0),
 (33, 1.0),
 (35, 1.0),
 (42, 1.0),
 (44, 1.0),
 (46, 1.0),
 (47, 1.0),
 (52, 1.0),
 (63, 1.0),
 (70, 1.0)]
\end{Verbatim}
\end{tcolorbox}
        
    \hypertarget{points-top-10-highest-prestige-centrality}{%
\subsection*{5. (3 points) Top 10 Highest Prestige
Centrality}\label{points-top-10-highest-prestige-centrality}}

    Find the top 10 nodes as ranked by prestige centrality (eigenvector
centrality in networkx).

    \begin{tcolorbox}[breakable, size=fbox, boxrule=1pt, pad at break*=1mm,colback=cellbackground, colframe=cellborder]
\prompt{In}{incolor}{ }{\boxspacing}
\begin{Verbatim}[commandchars=\\\{\}]
\PY{c+c1}{\PYZsh{} prestige centrality}
\PY{n}{eigenvector\PYZus{}centrality} \PY{o}{=} \PY{n}{nx}\PY{o}{.}\PY{n}{eigenvector\PYZus{}centrality}\PY{p}{(}\PY{n}{G}\PY{p}{)}
\PY{n}{topSortedDict}\PY{p}{(}\PY{n}{eigenvector\PYZus{}centrality}\PY{p}{)}
\end{Verbatim}
\end{tcolorbox}

            \begin{tcolorbox}[breakable, size=fbox, boxrule=.5pt, pad at break*=1mm, opacityfill=0]
\prompt{Out}{outcolor}{ }{\boxspacing}
\begin{Verbatim}[commandchars=\\\{\}]
\{1912: 0.09540696149067629,
 1993: 0.0835324284081597,
 2078: 0.08413617041724979,
 2123: 0.08367141238206226,
 2142: 0.08419311897991796,
 2206: 0.08605239270584343,
 2218: 0.08415573568055032,
 2233: 0.08517340912756598,
 2266: 0.08698327767886553,
 2464: 0.08427877475676092\}
\end{Verbatim}
\end{tcolorbox}
        
    \hypertarget{points-top-10-highest-ranked-by-pagerank}{%
\subsection*{6. (3 points) Top 10 Highest Ranked by
Pagerank}\label{points-top-10-highest-ranked-by-pagerank}}

    Find the top 10 nodes as ranked by Pagerank.

    \begin{tcolorbox}[breakable, size=fbox, boxrule=1pt, pad at break*=1mm,colback=cellbackground, colframe=cellborder]
\prompt{In}{incolor}{ }{\boxspacing}
\begin{Verbatim}[commandchars=\\\{\}]
\PY{n}{topSortedDict}\PY{p}{(}\PY{n}{nx}\PY{o}{.}\PY{n}{pagerank}\PY{p}{(}\PY{n}{G}\PY{p}{)}\PY{p}{)}
\end{Verbatim}
\end{tcolorbox}

            \begin{tcolorbox}[breakable, size=fbox, boxrule=.5pt, pad at break*=1mm, opacityfill=0]
\prompt{Out}{outcolor}{ }{\boxspacing}
\begin{Verbatim}[commandchars=\\\{\}]
\{0: 0.006289602618466542,
 107: 0.006936420955866117,
 348: 0.002348096972780577,
 414: 0.001800299047070226,
 686: 0.002219359259800019,
 698: 0.0013171153138368812,
 1684: 0.006367162138306824,
 1912: 0.003876971600884498,
 3437: 0.0076145868447496,
 3980: 0.0021703235790099928\}
\end{Verbatim}
\end{tcolorbox}
        
    \hypertarget{points-commet-on-diff.-and-sim.}{%
\subsection*{7. (3 points) Commet on Diff. and
Sim.}\label{points-commet-on-diff.-and-sim.}}

    Comment on the differences and similarities in questions Part 3 1-6. Are
the highly ranked nodes mostly the same? Do you notice significant
differences in the rankings? Why do you think this is the case?

    The highest degree a node had was 1045 and that was far from node 107.
Node 107 also had a very high betweenness centrality which is expected
as it has a large sum of paths leading to and from it. Node 107 was also
seen to have a large PageRank value. This was also expected since a node
that is pointed to repeatedly will have a higher PageRank and this
logically follows that a node with a high degree will have lots of other
nodes pointing to it. However, this node did not have a significant
clustering coefficient or a high Prestige centrality. Additionally, Node
1912 had a top 10 rank for PageRank, Betweenness centrality, and
Prestige centrality. This shows that there are multiple different nodes
that could have a perceived ``weight'' to them and should be considered
important. It is important to note that no node that ranked highly in
the clustering coefficient ranked highly in any other category so it
could be assumed that those nodes are very highly clustered but may not
be highly clustered with the rest of the graph.

    \hypertarget{tips-and-acknowledgements}{%
\section*{Tips and Acknowledgements}\label{tips-and-acknowledgements}}

Make sure to submit your answer as a PDF on Gradscope and Brightspace.
Make sure to show your work. Include any code snippets you used to
generate an answer, using comments in the code to clearly indicate which
problem corresponds to which code.

\textbf{Acknowledgements}: Project adapted from assignments of Veronika
Strnadova-Neeley.


    % Add a bibliography block to the postdoc
    
    
    
\end{document}
