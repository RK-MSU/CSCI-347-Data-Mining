\documentclass[11pt]{article}

    \usepackage[breakable]{tcolorbox}
    \usepackage{parskip} % Stop auto-indenting (to mimic markdown behaviour)
    

    % Basic figure setup, for now with no caption control since it's done
    % automatically by Pandoc (which extracts ![](path) syntax from Markdown).
    \usepackage{graphicx}
    % Maintain compatibility with old templates. Remove in nbconvert 6.0
    \let\Oldincludegraphics\includegraphics
    % Ensure that by default, figures have no caption (until we provide a
    % proper Figure object with a Caption API and a way to capture that
    % in the conversion process - todo).
    \usepackage{caption}
    \DeclareCaptionFormat{nocaption}{}
    \captionsetup{format=nocaption,aboveskip=0pt,belowskip=0pt}

    \usepackage{float}
    \floatplacement{figure}{H} % forces figures to be placed at the correct location
    \usepackage{xcolor} % Allow colors to be defined
    \usepackage{enumerate} % Needed for markdown enumerations to work
    \usepackage{geometry} % Used to adjust the document margins
    \usepackage{amsmath} % Equations
    \usepackage{amssymb} % Equations
    \usepackage{textcomp} % defines textquotesingle
    % Hack from http://tex.stackexchange.com/a/47451/13684:
    \AtBeginDocument{%
        \def\PYZsq{\textquotesingle}% Upright quotes in Pygmentized code
    }
    \usepackage{upquote} % Upright quotes for verbatim code
    \usepackage{eurosym} % defines \euro

    \usepackage{iftex}
    \ifPDFTeX
        \usepackage[T1]{fontenc}
        \IfFileExists{alphabeta.sty}{
              \usepackage{alphabeta}
          }{
              \usepackage[mathletters]{ucs}
              \usepackage[utf8x]{inputenc}
          }
    \else
        \usepackage{fontspec}
        \usepackage{unicode-math}
    \fi

    \usepackage{fancyvrb} % verbatim replacement that allows latex
    \usepackage{grffile} % extends the file name processing of package graphics 
                         % to support a larger range
    \makeatletter % fix for old versions of grffile with XeLaTeX
    \@ifpackagelater{grffile}{2019/11/01}
    {
      % Do nothing on new versions
    }
    {
      \def\Gread@@xetex#1{%
        \IfFileExists{"\Gin@base".bb}%
        {\Gread@eps{\Gin@base.bb}}%
        {\Gread@@xetex@aux#1}%
      }
    }
    \makeatother
    \usepackage[Export]{adjustbox} % Used to constrain images to a maximum size
    \adjustboxset{max size={0.9\linewidth}{0.9\paperheight}}

    % The hyperref package gives us a pdf with properly built
    % internal navigation ('pdf bookmarks' for the table of contents,
    % internal cross-reference links, web links for URLs, etc.)
    \usepackage{hyperref}
    % The default LaTeX title has an obnoxious amount of whitespace. By default,
    % titling removes some of it. It also provides customization options.
    \usepackage{titling}
    \usepackage{longtable} % longtable support required by pandoc >1.10
    \usepackage{booktabs}  % table support for pandoc > 1.12.2
    \usepackage{array}     % table support for pandoc >= 2.11.3
    \usepackage{calc}      % table minipage width calculation for pandoc >= 2.11.1
    \usepackage[inline]{enumitem} % IRkernel/repr support (it uses the enumerate* environment)
    \usepackage[normalem]{ulem} % ulem is needed to support strikethroughs (\sout)
                                % normalem makes italics be italics, not underlines
    \usepackage{mathrsfs}
    

    
    % Colors for the hyperref package
    \definecolor{urlcolor}{rgb}{0,.145,.698}
    \definecolor{linkcolor}{rgb}{.71,0.21,0.01}
    \definecolor{citecolor}{rgb}{.12,.54,.11}

    % ANSI colors
    \definecolor{ansi-black}{HTML}{3E424D}
    \definecolor{ansi-black-intense}{HTML}{282C36}
    \definecolor{ansi-red}{HTML}{E75C58}
    \definecolor{ansi-red-intense}{HTML}{B22B31}
    \definecolor{ansi-green}{HTML}{00A250}
    \definecolor{ansi-green-intense}{HTML}{007427}
    \definecolor{ansi-yellow}{HTML}{DDB62B}
    \definecolor{ansi-yellow-intense}{HTML}{B27D12}
    \definecolor{ansi-blue}{HTML}{208FFB}
    \definecolor{ansi-blue-intense}{HTML}{0065CA}
    \definecolor{ansi-magenta}{HTML}{D160C4}
    \definecolor{ansi-magenta-intense}{HTML}{A03196}
    \definecolor{ansi-cyan}{HTML}{60C6C8}
    \definecolor{ansi-cyan-intense}{HTML}{258F8F}
    \definecolor{ansi-white}{HTML}{C5C1B4}
    \definecolor{ansi-white-intense}{HTML}{A1A6B2}
    \definecolor{ansi-default-inverse-fg}{HTML}{FFFFFF}
    \definecolor{ansi-default-inverse-bg}{HTML}{000000}

    % common color for the border for error outputs.
    \definecolor{outerrorbackground}{HTML}{FFDFDF}

    % commands and environments needed by pandoc snippets
    % extracted from the output of `pandoc -s`
    \providecommand{\tightlist}{%
      \setlength{\itemsep}{0pt}\setlength{\parskip}{0pt}}
    \DefineVerbatimEnvironment{Highlighting}{Verbatim}{commandchars=\\\{\}}
    % Add ',fontsize=\small' for more characters per line
    \newenvironment{Shaded}{}{}
    \newcommand{\KeywordTok}[1]{\textcolor[rgb]{0.00,0.44,0.13}{\textbf{{#1}}}}
    \newcommand{\DataTypeTok}[1]{\textcolor[rgb]{0.56,0.13,0.00}{{#1}}}
    \newcommand{\DecValTok}[1]{\textcolor[rgb]{0.25,0.63,0.44}{{#1}}}
    \newcommand{\BaseNTok}[1]{\textcolor[rgb]{0.25,0.63,0.44}{{#1}}}
    \newcommand{\FloatTok}[1]{\textcolor[rgb]{0.25,0.63,0.44}{{#1}}}
    \newcommand{\CharTok}[1]{\textcolor[rgb]{0.25,0.44,0.63}{{#1}}}
    \newcommand{\StringTok}[1]{\textcolor[rgb]{0.25,0.44,0.63}{{#1}}}
    \newcommand{\CommentTok}[1]{\textcolor[rgb]{0.38,0.63,0.69}{\textit{{#1}}}}
    \newcommand{\OtherTok}[1]{\textcolor[rgb]{0.00,0.44,0.13}{{#1}}}
    \newcommand{\AlertTok}[1]{\textcolor[rgb]{1.00,0.00,0.00}{\textbf{{#1}}}}
    \newcommand{\FunctionTok}[1]{\textcolor[rgb]{0.02,0.16,0.49}{{#1}}}
    \newcommand{\RegionMarkerTok}[1]{{#1}}
    \newcommand{\ErrorTok}[1]{\textcolor[rgb]{1.00,0.00,0.00}{\textbf{{#1}}}}
    \newcommand{\NormalTok}[1]{{#1}}
    
    % Additional commands for more recent versions of Pandoc
    \newcommand{\ConstantTok}[1]{\textcolor[rgb]{0.53,0.00,0.00}{{#1}}}
    \newcommand{\SpecialCharTok}[1]{\textcolor[rgb]{0.25,0.44,0.63}{{#1}}}
    \newcommand{\VerbatimStringTok}[1]{\textcolor[rgb]{0.25,0.44,0.63}{{#1}}}
    \newcommand{\SpecialStringTok}[1]{\textcolor[rgb]{0.73,0.40,0.53}{{#1}}}
    \newcommand{\ImportTok}[1]{{#1}}
    \newcommand{\DocumentationTok}[1]{\textcolor[rgb]{0.73,0.13,0.13}{\textit{{#1}}}}
    \newcommand{\AnnotationTok}[1]{\textcolor[rgb]{0.38,0.63,0.69}{\textbf{\textit{{#1}}}}}
    \newcommand{\CommentVarTok}[1]{\textcolor[rgb]{0.38,0.63,0.69}{\textbf{\textit{{#1}}}}}
    \newcommand{\VariableTok}[1]{\textcolor[rgb]{0.10,0.09,0.49}{{#1}}}
    \newcommand{\ControlFlowTok}[1]{\textcolor[rgb]{0.00,0.44,0.13}{\textbf{{#1}}}}
    \newcommand{\OperatorTok}[1]{\textcolor[rgb]{0.40,0.40,0.40}{{#1}}}
    \newcommand{\BuiltInTok}[1]{{#1}}
    \newcommand{\ExtensionTok}[1]{{#1}}
    \newcommand{\PreprocessorTok}[1]{\textcolor[rgb]{0.74,0.48,0.00}{{#1}}}
    \newcommand{\AttributeTok}[1]{\textcolor[rgb]{0.49,0.56,0.16}{{#1}}}
    \newcommand{\InformationTok}[1]{\textcolor[rgb]{0.38,0.63,0.69}{\textbf{\textit{{#1}}}}}
    \newcommand{\WarningTok}[1]{\textcolor[rgb]{0.38,0.63,0.69}{\textbf{\textit{{#1}}}}}
    
    
    % Define a nice break command that doesn't care if a line doesn't already
    % exist.
    \def\br{\hspace*{\fill} \\* }
    % Math Jax compatibility definitions
    \def\gt{>}
    \def\lt{<}
    \let\Oldtex\TeX
    \let\Oldlatex\LaTeX
    \renewcommand{\TeX}{\textrm{\Oldtex}}
    \renewcommand{\LaTeX}{\textrm{\Oldlatex}}
    % Document parameters
    % Document title
    \title{final-project}
    
    
    
    
    
% Pygments definitions
\makeatletter
\def\PY@reset{\let\PY@it=\relax \let\PY@bf=\relax%
    \let\PY@ul=\relax \let\PY@tc=\relax%
    \let\PY@bc=\relax \let\PY@ff=\relax}
\def\PY@tok#1{\csname PY@tok@#1\endcsname}
\def\PY@toks#1+{\ifx\relax#1\empty\else%
    \PY@tok{#1}\expandafter\PY@toks\fi}
\def\PY@do#1{\PY@bc{\PY@tc{\PY@ul{%
    \PY@it{\PY@bf{\PY@ff{#1}}}}}}}
\def\PY#1#2{\PY@reset\PY@toks#1+\relax+\PY@do{#2}}

\@namedef{PY@tok@w}{\def\PY@tc##1{\textcolor[rgb]{0.73,0.73,0.73}{##1}}}
\@namedef{PY@tok@c}{\let\PY@it=\textit\def\PY@tc##1{\textcolor[rgb]{0.24,0.48,0.48}{##1}}}
\@namedef{PY@tok@cp}{\def\PY@tc##1{\textcolor[rgb]{0.61,0.40,0.00}{##1}}}
\@namedef{PY@tok@k}{\let\PY@bf=\textbf\def\PY@tc##1{\textcolor[rgb]{0.00,0.50,0.00}{##1}}}
\@namedef{PY@tok@kp}{\def\PY@tc##1{\textcolor[rgb]{0.00,0.50,0.00}{##1}}}
\@namedef{PY@tok@kt}{\def\PY@tc##1{\textcolor[rgb]{0.69,0.00,0.25}{##1}}}
\@namedef{PY@tok@o}{\def\PY@tc##1{\textcolor[rgb]{0.40,0.40,0.40}{##1}}}
\@namedef{PY@tok@ow}{\let\PY@bf=\textbf\def\PY@tc##1{\textcolor[rgb]{0.67,0.13,1.00}{##1}}}
\@namedef{PY@tok@nb}{\def\PY@tc##1{\textcolor[rgb]{0.00,0.50,0.00}{##1}}}
\@namedef{PY@tok@nf}{\def\PY@tc##1{\textcolor[rgb]{0.00,0.00,1.00}{##1}}}
\@namedef{PY@tok@nc}{\let\PY@bf=\textbf\def\PY@tc##1{\textcolor[rgb]{0.00,0.00,1.00}{##1}}}
\@namedef{PY@tok@nn}{\let\PY@bf=\textbf\def\PY@tc##1{\textcolor[rgb]{0.00,0.00,1.00}{##1}}}
\@namedef{PY@tok@ne}{\let\PY@bf=\textbf\def\PY@tc##1{\textcolor[rgb]{0.80,0.25,0.22}{##1}}}
\@namedef{PY@tok@nv}{\def\PY@tc##1{\textcolor[rgb]{0.10,0.09,0.49}{##1}}}
\@namedef{PY@tok@no}{\def\PY@tc##1{\textcolor[rgb]{0.53,0.00,0.00}{##1}}}
\@namedef{PY@tok@nl}{\def\PY@tc##1{\textcolor[rgb]{0.46,0.46,0.00}{##1}}}
\@namedef{PY@tok@ni}{\let\PY@bf=\textbf\def\PY@tc##1{\textcolor[rgb]{0.44,0.44,0.44}{##1}}}
\@namedef{PY@tok@na}{\def\PY@tc##1{\textcolor[rgb]{0.41,0.47,0.13}{##1}}}
\@namedef{PY@tok@nt}{\let\PY@bf=\textbf\def\PY@tc##1{\textcolor[rgb]{0.00,0.50,0.00}{##1}}}
\@namedef{PY@tok@nd}{\def\PY@tc##1{\textcolor[rgb]{0.67,0.13,1.00}{##1}}}
\@namedef{PY@tok@s}{\def\PY@tc##1{\textcolor[rgb]{0.73,0.13,0.13}{##1}}}
\@namedef{PY@tok@sd}{\let\PY@it=\textit\def\PY@tc##1{\textcolor[rgb]{0.73,0.13,0.13}{##1}}}
\@namedef{PY@tok@si}{\let\PY@bf=\textbf\def\PY@tc##1{\textcolor[rgb]{0.64,0.35,0.47}{##1}}}
\@namedef{PY@tok@se}{\let\PY@bf=\textbf\def\PY@tc##1{\textcolor[rgb]{0.67,0.36,0.12}{##1}}}
\@namedef{PY@tok@sr}{\def\PY@tc##1{\textcolor[rgb]{0.64,0.35,0.47}{##1}}}
\@namedef{PY@tok@ss}{\def\PY@tc##1{\textcolor[rgb]{0.10,0.09,0.49}{##1}}}
\@namedef{PY@tok@sx}{\def\PY@tc##1{\textcolor[rgb]{0.00,0.50,0.00}{##1}}}
\@namedef{PY@tok@m}{\def\PY@tc##1{\textcolor[rgb]{0.40,0.40,0.40}{##1}}}
\@namedef{PY@tok@gh}{\let\PY@bf=\textbf\def\PY@tc##1{\textcolor[rgb]{0.00,0.00,0.50}{##1}}}
\@namedef{PY@tok@gu}{\let\PY@bf=\textbf\def\PY@tc##1{\textcolor[rgb]{0.50,0.00,0.50}{##1}}}
\@namedef{PY@tok@gd}{\def\PY@tc##1{\textcolor[rgb]{0.63,0.00,0.00}{##1}}}
\@namedef{PY@tok@gi}{\def\PY@tc##1{\textcolor[rgb]{0.00,0.52,0.00}{##1}}}
\@namedef{PY@tok@gr}{\def\PY@tc##1{\textcolor[rgb]{0.89,0.00,0.00}{##1}}}
\@namedef{PY@tok@ge}{\let\PY@it=\textit}
\@namedef{PY@tok@gs}{\let\PY@bf=\textbf}
\@namedef{PY@tok@gp}{\let\PY@bf=\textbf\def\PY@tc##1{\textcolor[rgb]{0.00,0.00,0.50}{##1}}}
\@namedef{PY@tok@go}{\def\PY@tc##1{\textcolor[rgb]{0.44,0.44,0.44}{##1}}}
\@namedef{PY@tok@gt}{\def\PY@tc##1{\textcolor[rgb]{0.00,0.27,0.87}{##1}}}
\@namedef{PY@tok@err}{\def\PY@bc##1{{\setlength{\fboxsep}{\string -\fboxrule}\fcolorbox[rgb]{1.00,0.00,0.00}{1,1,1}{\strut ##1}}}}
\@namedef{PY@tok@kc}{\let\PY@bf=\textbf\def\PY@tc##1{\textcolor[rgb]{0.00,0.50,0.00}{##1}}}
\@namedef{PY@tok@kd}{\let\PY@bf=\textbf\def\PY@tc##1{\textcolor[rgb]{0.00,0.50,0.00}{##1}}}
\@namedef{PY@tok@kn}{\let\PY@bf=\textbf\def\PY@tc##1{\textcolor[rgb]{0.00,0.50,0.00}{##1}}}
\@namedef{PY@tok@kr}{\let\PY@bf=\textbf\def\PY@tc##1{\textcolor[rgb]{0.00,0.50,0.00}{##1}}}
\@namedef{PY@tok@bp}{\def\PY@tc##1{\textcolor[rgb]{0.00,0.50,0.00}{##1}}}
\@namedef{PY@tok@fm}{\def\PY@tc##1{\textcolor[rgb]{0.00,0.00,1.00}{##1}}}
\@namedef{PY@tok@vc}{\def\PY@tc##1{\textcolor[rgb]{0.10,0.09,0.49}{##1}}}
\@namedef{PY@tok@vg}{\def\PY@tc##1{\textcolor[rgb]{0.10,0.09,0.49}{##1}}}
\@namedef{PY@tok@vi}{\def\PY@tc##1{\textcolor[rgb]{0.10,0.09,0.49}{##1}}}
\@namedef{PY@tok@vm}{\def\PY@tc##1{\textcolor[rgb]{0.10,0.09,0.49}{##1}}}
\@namedef{PY@tok@sa}{\def\PY@tc##1{\textcolor[rgb]{0.73,0.13,0.13}{##1}}}
\@namedef{PY@tok@sb}{\def\PY@tc##1{\textcolor[rgb]{0.73,0.13,0.13}{##1}}}
\@namedef{PY@tok@sc}{\def\PY@tc##1{\textcolor[rgb]{0.73,0.13,0.13}{##1}}}
\@namedef{PY@tok@dl}{\def\PY@tc##1{\textcolor[rgb]{0.73,0.13,0.13}{##1}}}
\@namedef{PY@tok@s2}{\def\PY@tc##1{\textcolor[rgb]{0.73,0.13,0.13}{##1}}}
\@namedef{PY@tok@sh}{\def\PY@tc##1{\textcolor[rgb]{0.73,0.13,0.13}{##1}}}
\@namedef{PY@tok@s1}{\def\PY@tc##1{\textcolor[rgb]{0.73,0.13,0.13}{##1}}}
\@namedef{PY@tok@mb}{\def\PY@tc##1{\textcolor[rgb]{0.40,0.40,0.40}{##1}}}
\@namedef{PY@tok@mf}{\def\PY@tc##1{\textcolor[rgb]{0.40,0.40,0.40}{##1}}}
\@namedef{PY@tok@mh}{\def\PY@tc##1{\textcolor[rgb]{0.40,0.40,0.40}{##1}}}
\@namedef{PY@tok@mi}{\def\PY@tc##1{\textcolor[rgb]{0.40,0.40,0.40}{##1}}}
\@namedef{PY@tok@il}{\def\PY@tc##1{\textcolor[rgb]{0.40,0.40,0.40}{##1}}}
\@namedef{PY@tok@mo}{\def\PY@tc##1{\textcolor[rgb]{0.40,0.40,0.40}{##1}}}
\@namedef{PY@tok@ch}{\let\PY@it=\textit\def\PY@tc##1{\textcolor[rgb]{0.24,0.48,0.48}{##1}}}
\@namedef{PY@tok@cm}{\let\PY@it=\textit\def\PY@tc##1{\textcolor[rgb]{0.24,0.48,0.48}{##1}}}
\@namedef{PY@tok@cpf}{\let\PY@it=\textit\def\PY@tc##1{\textcolor[rgb]{0.24,0.48,0.48}{##1}}}
\@namedef{PY@tok@c1}{\let\PY@it=\textit\def\PY@tc##1{\textcolor[rgb]{0.24,0.48,0.48}{##1}}}
\@namedef{PY@tok@cs}{\let\PY@it=\textit\def\PY@tc##1{\textcolor[rgb]{0.24,0.48,0.48}{##1}}}

\def\PYZbs{\char`\\}
\def\PYZus{\char`\_}
\def\PYZob{\char`\{}
\def\PYZcb{\char`\}}
\def\PYZca{\char`\^}
\def\PYZam{\char`\&}
\def\PYZlt{\char`\<}
\def\PYZgt{\char`\>}
\def\PYZsh{\char`\#}
\def\PYZpc{\char`\%}
\def\PYZdl{\char`\$}
\def\PYZhy{\char`\-}
\def\PYZsq{\char`\'}
\def\PYZdq{\char`\"}
\def\PYZti{\char`\~}
% for compatibility with earlier versions
\def\PYZat{@}
\def\PYZlb{[}
\def\PYZrb{]}
\makeatother


    % For linebreaks inside Verbatim environment from package fancyvrb. 
    \makeatletter
        \newbox\Wrappedcontinuationbox 
        \newbox\Wrappedvisiblespacebox 
        \newcommand*\Wrappedvisiblespace {\textcolor{red}{\textvisiblespace}} 
        \newcommand*\Wrappedcontinuationsymbol {\textcolor{red}{\llap{\tiny$\m@th\hookrightarrow$}}} 
        \newcommand*\Wrappedcontinuationindent {3ex } 
        \newcommand*\Wrappedafterbreak {\kern\Wrappedcontinuationindent\copy\Wrappedcontinuationbox} 
        % Take advantage of the already applied Pygments mark-up to insert 
        % potential linebreaks for TeX processing. 
        %        {, <, #, %, $, ' and ": go to next line. 
        %        _, }, ^, &, >, - and ~: stay at end of broken line. 
        % Use of \textquotesingle for straight quote. 
        \newcommand*\Wrappedbreaksatspecials {% 
            \def\PYGZus{\discretionary{\char`\_}{\Wrappedafterbreak}{\char`\_}}% 
            \def\PYGZob{\discretionary{}{\Wrappedafterbreak\char`\{}{\char`\{}}% 
            \def\PYGZcb{\discretionary{\char`\}}{\Wrappedafterbreak}{\char`\}}}% 
            \def\PYGZca{\discretionary{\char`\^}{\Wrappedafterbreak}{\char`\^}}% 
            \def\PYGZam{\discretionary{\char`\&}{\Wrappedafterbreak}{\char`\&}}% 
            \def\PYGZlt{\discretionary{}{\Wrappedafterbreak\char`\<}{\char`\<}}% 
            \def\PYGZgt{\discretionary{\char`\>}{\Wrappedafterbreak}{\char`\>}}% 
            \def\PYGZsh{\discretionary{}{\Wrappedafterbreak\char`\#}{\char`\#}}% 
            \def\PYGZpc{\discretionary{}{\Wrappedafterbreak\char`\%}{\char`\%}}% 
            \def\PYGZdl{\discretionary{}{\Wrappedafterbreak\char`\$}{\char`\$}}% 
            \def\PYGZhy{\discretionary{\char`\-}{\Wrappedafterbreak}{\char`\-}}% 
            \def\PYGZsq{\discretionary{}{\Wrappedafterbreak\textquotesingle}{\textquotesingle}}% 
            \def\PYGZdq{\discretionary{}{\Wrappedafterbreak\char`\"}{\char`\"}}% 
            \def\PYGZti{\discretionary{\char`\~}{\Wrappedafterbreak}{\char`\~}}% 
        } 
        % Some characters . , ; ? ! / are not pygmentized. 
        % This macro makes them "active" and they will insert potential linebreaks 
        \newcommand*\Wrappedbreaksatpunct {% 
            \lccode`\~`\.\lowercase{\def~}{\discretionary{\hbox{\char`\.}}{\Wrappedafterbreak}{\hbox{\char`\.}}}% 
            \lccode`\~`\,\lowercase{\def~}{\discretionary{\hbox{\char`\,}}{\Wrappedafterbreak}{\hbox{\char`\,}}}% 
            \lccode`\~`\;\lowercase{\def~}{\discretionary{\hbox{\char`\;}}{\Wrappedafterbreak}{\hbox{\char`\;}}}% 
            \lccode`\~`\:\lowercase{\def~}{\discretionary{\hbox{\char`\:}}{\Wrappedafterbreak}{\hbox{\char`\:}}}% 
            \lccode`\~`\?\lowercase{\def~}{\discretionary{\hbox{\char`\?}}{\Wrappedafterbreak}{\hbox{\char`\?}}}% 
            \lccode`\~`\!\lowercase{\def~}{\discretionary{\hbox{\char`\!}}{\Wrappedafterbreak}{\hbox{\char`\!}}}% 
            \lccode`\~`\/\lowercase{\def~}{\discretionary{\hbox{\char`\/}}{\Wrappedafterbreak}{\hbox{\char`\/}}}% 
            \catcode`\.\active
            \catcode`\,\active 
            \catcode`\;\active
            \catcode`\:\active
            \catcode`\?\active
            \catcode`\!\active
            \catcode`\/\active 
            \lccode`\~`\~ 	
        }
    \makeatother

    \let\OriginalVerbatim=\Verbatim
    \makeatletter
    \renewcommand{\Verbatim}[1][1]{%
        %\parskip\z@skip
        \sbox\Wrappedcontinuationbox {\Wrappedcontinuationsymbol}%
        \sbox\Wrappedvisiblespacebox {\FV@SetupFont\Wrappedvisiblespace}%
        \def\FancyVerbFormatLine ##1{\hsize\linewidth
            \vtop{\raggedright\hyphenpenalty\z@\exhyphenpenalty\z@
                \doublehyphendemerits\z@\finalhyphendemerits\z@
                \strut ##1\strut}%
        }%
        % If the linebreak is at a space, the latter will be displayed as visible
        % space at end of first line, and a continuation symbol starts next line.
        % Stretch/shrink are however usually zero for typewriter font.
        \def\FV@Space {%
            \nobreak\hskip\z@ plus\fontdimen3\font minus\fontdimen4\font
            \discretionary{\copy\Wrappedvisiblespacebox}{\Wrappedafterbreak}
            {\kern\fontdimen2\font}%
        }%
        
        % Allow breaks at special characters using \PYG... macros.
        \Wrappedbreaksatspecials
        % Breaks at punctuation characters . , ; ? ! and / need catcode=\active 	
        \OriginalVerbatim[#1,codes*=\Wrappedbreaksatpunct]%
    }
    \makeatother

    % Exact colors from NB
    \definecolor{incolor}{HTML}{303F9F}
    \definecolor{outcolor}{HTML}{D84315}
    \definecolor{cellborder}{HTML}{CFCFCF}
    \definecolor{cellbackground}{HTML}{F7F7F7}
    
    % prompt
    \makeatletter
    \newcommand{\boxspacing}{\kern\kvtcb@left@rule\kern\kvtcb@boxsep}
    \makeatother
    \newcommand{\prompt}[4]{
        {\ttfamily\llap{{\color{#2}[#3]:\hspace{3pt}#4}}\vspace{-\baselineskip}}
    }
    

    
    % Prevent overflowing lines due to hard-to-break entities
    \sloppy 
    % Setup hyperref package
    \hypersetup{
      breaklinks=true,  % so long urls are correctly broken across lines
      colorlinks=true,
      urlcolor=urlcolor,
      linkcolor=linkcolor,
      citecolor=citecolor,
      }
    % Slightly bigger margins than the latex defaults
    
    \geometry{verbose,tmargin=1in,bmargin=1in,lmargin=1in,rmargin=1in}
    
    

\begin{document}
    
    \maketitle
    
    

    
    River Kelly

Kyler Gappa

CSCI-347: Data Mining

Final Project

    \hypertarget{part-1-plan-20-points}{%
\section*{Part 1: Plan (20 points)}\label{part-1-plan-20-points}}

    The problem that we are investigating is the association between
developed liver disorders and the possible contributing factors that
lead to such a diagnosis. It is no question that the consumption of
alcoholic beverages increases the chances of developing a liver
disorder, but the question still remains why some people are more at
risk than others. In addition to alcohol consumption, other compounds
found in the liver have been found to have confounding associations.

The data set that we have selected is the
\href{https://archive.ics.uci.edu/ml/datasets/Liver+Disorders}{Liver
Disorders Data Set} (Forsyth). The data set includes a total of seven
attributes. The first five attributes are numerical values pulled from
the blood test of the patient. Since these values are thought to be
sensitive to liver disease, they may show common traits so that early
detection is more viable. The next variable is the number of drinks of
alcohol per day. This could be a connecting factor for some of the test
results. The final attribute is a categorical variable to help separate
the data into test or training sets.

The data mining techniques that we will use to solve this problem
include dimensionality reduction. There exists one categorical attribute
within our data set that will need to be removed as it only contains
information for if it was intended for a training dataset. We plan to
use dimensionality reduction across all of the provided attributes in an
attempt to identify which attributes have the greatest effect.

If we run out of time we plan to not explore the differences in the
training and test categories to reduce the scope of the project from two
separate datasets to a single dataset. If this were to happen we would
totally ignore the categorical classification and run all testing over
the entire dataset as a whole.

    \hypertarget{part-2-implement-30-points}{%
\section*{Part 2: Implement (30
points)}\label{part-2-implement-30-points}}

    \hypertarget{setup-code}{%
\subsection*{Setup Code}\label{setup-code}}

    \hypertarget{import-libraries}{%
\subsubsection*{Import Libraries}\label{import-libraries}}

    \begin{tcolorbox}[breakable, size=fbox, boxrule=1pt, pad at break*=1mm,colback=cellbackground, colframe=cellborder]
\prompt{In}{incolor}{5}{\boxspacing}
\begin{Verbatim}[commandchars=\\\{\}]
\PY{k+kn}{import} \PY{n+nn}{numpy} \PY{k}{as} \PY{n+nn}{np}
\PY{k+kn}{import} \PY{n+nn}{pandas} \PY{k}{as} \PY{n+nn}{pd}
\PY{k+kn}{import} \PY{n+nn}{matplotlib}\PY{n+nn}{.}\PY{n+nn}{pyplot} \PY{k}{as} \PY{n+nn}{plt}
\PY{k+kn}{from} \PY{n+nn}{sklearn}\PY{n+nn}{.}\PY{n+nn}{decomposition} \PY{k+kn}{import} \PY{n}{PCA}
\PY{k+kn}{from} \PY{n+nn}{google}\PY{n+nn}{.}\PY{n+nn}{colab} \PY{k+kn}{import} \PY{n}{drive}
\PY{k+kn}{import} \PY{n+nn}{numpy}\PY{n+nn}{.}\PY{n+nn}{linalg} \PY{k}{as} \PY{n+nn}{LA}
\PY{k+kn}{import} \PY{n+nn}{os}
\end{Verbatim}
\end{tcolorbox}

    \hypertarget{global-const.-variables}{%
\subsubsection*{Global (Const.)
Variables}\label{global-const.-variables}}

    \begin{tcolorbox}[breakable, size=fbox, boxrule=1pt, pad at break*=1mm,colback=cellbackground, colframe=cellborder]
\prompt{In}{incolor}{6}{\boxspacing}
\begin{Verbatim}[commandchars=\\\{\}]
\PY{c+c1}{\PYZsh{} file path for the location of the data file}
\PY{n}{DATA\PYZus{}FILENAME} \PY{o}{=} \PY{l+s+s2}{\PYZdq{}}\PY{l+s+s2}{/content/drive/My Drive/347\PYZhy{}Data\PYZhy{}Mining/Final\PYZhy{}Project/data/bupa.data}\PY{l+s+s2}{\PYZdq{}}

\PY{c+c1}{\PYZsh{} dictionary of columns for the data}
\PY{n}{DATA\PYZus{}COLUMN\PYZus{}NAMES} \PY{o}{=} \PY{p}{\PYZob{}}
    \PY{l+s+s1}{\PYZsq{}}\PY{l+s+s1}{mcv}\PY{l+s+s1}{\PYZsq{}}\PY{p}{:} \PY{l+s+s1}{\PYZsq{}}\PY{l+s+s1}{Mean Corpuscular Volume}\PY{l+s+s1}{\PYZsq{}}\PY{p}{,}
    \PY{l+s+s1}{\PYZsq{}}\PY{l+s+s1}{alkphos}\PY{l+s+s1}{\PYZsq{}}\PY{p}{:} \PY{l+s+s1}{\PYZsq{}}\PY{l+s+s1}{Alkaline Phosphotase}\PY{l+s+s1}{\PYZsq{}}\PY{p}{,}
    \PY{l+s+s1}{\PYZsq{}}\PY{l+s+s1}{sgpt}\PY{l+s+s1}{\PYZsq{}}\PY{p}{:} \PY{l+s+s1}{\PYZsq{}}\PY{l+s+s1}{Alamine Aminotransferase}\PY{l+s+s1}{\PYZsq{}}\PY{p}{,}
    \PY{l+s+s1}{\PYZsq{}}\PY{l+s+s1}{sgot}\PY{l+s+s1}{\PYZsq{}}\PY{p}{:} \PY{l+s+s1}{\PYZsq{}}\PY{l+s+s1}{Aspartate Aminotransferase}\PY{l+s+s1}{\PYZsq{}}\PY{p}{,}
    \PY{l+s+s1}{\PYZsq{}}\PY{l+s+s1}{gmmagt}\PY{l+s+s1}{\PYZsq{}}\PY{p}{:} \PY{l+s+s1}{\PYZsq{}}\PY{l+s+s1}{Gamma\PYZhy{}Glutamyl Transpeptidase}\PY{l+s+s1}{\PYZsq{}}\PY{p}{,}
    \PY{l+s+s1}{\PYZsq{}}\PY{l+s+s1}{drinks}\PY{l+s+s1}{\PYZsq{}}\PY{p}{:} \PY{l+s+s1}{\PYZsq{}}\PY{l+s+s1}{Number of half\PYZhy{}pint equivalents of alcoholic beverages drunk per day}\PY{l+s+s1}{\PYZsq{}}\PY{p}{,}
    \PY{c+c1}{\PYZsh{} \PYZsq{}selector\PYZsq{}: \PYZsq{}Group\PYZsq{} \PYZsh{} field used to split data into two sets}
\PY{p}{\PYZcb{}}

\PY{n}{DATA\PYZus{}COL\PYZus{}VALS} \PY{o}{=} \PY{n+nb}{list}\PY{p}{(}\PY{n}{DATA\PYZus{}COLUMN\PYZus{}NAMES}\PY{o}{.}\PY{n}{values}\PY{p}{(}\PY{p}{)}\PY{p}{)}

\PY{n}{dot\PYZus{}colors} \PY{o}{=} \PY{p}{[}\PY{l+s+s1}{\PYZsq{}}\PY{l+s+s1}{red}\PY{l+s+s1}{\PYZsq{}}\PY{p}{,} \PY{l+s+s1}{\PYZsq{}}\PY{l+s+s1}{orange}\PY{l+s+s1}{\PYZsq{}}\PY{p}{,} \PY{l+s+s1}{\PYZsq{}}\PY{l+s+s1}{blue}\PY{l+s+s1}{\PYZsq{}}\PY{p}{,} \PY{l+s+s1}{\PYZsq{}}\PY{l+s+s1}{black}\PY{l+s+s1}{\PYZsq{}}\PY{p}{,} \PY{l+s+s1}{\PYZsq{}}\PY{l+s+s1}{green}\PY{l+s+s1}{\PYZsq{}}\PY{p}{]}
\end{Verbatim}
\end{tcolorbox}

    \hypertarget{data-import-function}{%
\subsubsection*{Data Import Function}\label{data-import-function}}

    The following function is used read the data's file and parse the
contents.

    \begin{tcolorbox}[breakable, size=fbox, boxrule=1pt, pad at break*=1mm,colback=cellbackground, colframe=cellborder]
\prompt{In}{incolor}{7}{\boxspacing}
\begin{Verbatim}[commandchars=\\\{\}]
\PY{c+c1}{\PYZsh{} GetRawDataFromFile()}
\PY{c+c1}{\PYZsh{} returns a 2\PYZhy{}dimensional array of the data}
\PY{k}{def} \PY{n+nf}{getRawDataFromFile}\PY{p}{(}\PY{n}{file}\PY{p}{:} \PY{n+nb}{str}\PY{p}{)} \PY{o}{\PYZhy{}}\PY{o}{\PYZgt{}} \PY{n+nb}{list}\PY{p}{:}

    \PY{c+c1}{\PYZsh{} mount to google drive}
    \PY{n}{drive}\PY{o}{.}\PY{n}{mount}\PY{p}{(}\PY{l+s+s1}{\PYZsq{}}\PY{l+s+s1}{/content/drive}\PY{l+s+s1}{\PYZsq{}}\PY{p}{)}

    \PY{c+c1}{\PYZsh{} check that file exists}
    \PY{k}{if} \PY{o+ow}{not} \PY{n}{os}\PY{o}{.}\PY{n}{path}\PY{o}{.}\PY{n}{exists}\PY{p}{(}\PY{n}{file}\PY{p}{)}\PY{p}{:}
        \PY{c+c1}{\PYZsh{} file does NOT exist, raise error}
        \PY{k}{raise} \PY{n+ne}{RuntimeError}\PY{p}{(}\PY{l+s+sa}{f}\PY{l+s+s1}{\PYZsq{}}\PY{l+s+s1}{File }\PY{l+s+s1}{\PYZdq{}}\PY{l+s+si}{\PYZob{}}\PY{n}{file}\PY{l+s+si}{\PYZcb{}}\PY{l+s+s1}{\PYZdq{}}\PY{l+s+s1}{ does not exist}\PY{l+s+s1}{\PYZsq{}}\PY{p}{)}
    
    \PY{c+c1}{\PYZsh{} read file\PYZsq{}s lines}
    \PY{n}{lines} \PY{o}{=} \PY{n+nb}{list}\PY{p}{(}\PY{p}{)}
    \PY{k}{with} \PY{n+nb}{open}\PY{p}{(}\PY{n}{file}\PY{p}{,} \PY{l+s+s1}{\PYZsq{}}\PY{l+s+s1}{r}\PY{l+s+s1}{\PYZsq{}}\PY{p}{)} \PY{k}{as} \PY{n}{f}\PY{p}{:} \PY{n}{lines} \PY{o}{=} \PY{n}{f}\PY{o}{.}\PY{n}{readlines}\PY{p}{(}\PY{p}{)}

    \PY{c+c1}{\PYZsh{} unmount google drive}
    \PY{n}{drive}\PY{o}{.}\PY{n}{flush\PYZus{}and\PYZus{}unmount}\PY{p}{(}\PY{p}{)}

    \PY{c+c1}{\PYZsh{} parse the lines content and create the 2\PYZhy{}dimensioal array}
    \PY{n}{data} \PY{o}{=} \PY{n+nb}{list}\PY{p}{(}\PY{p}{)}
    \PY{c+c1}{\PYZsh{} iterate through each line}
    \PY{k}{for} \PY{n}{line} \PY{o+ow}{in} \PY{n}{lines}\PY{p}{:} 
        \PY{c+c1}{\PYZsh{} clean the line string}
        \PY{n}{line} \PY{o}{=} \PY{n+nb}{str}\PY{p}{(}\PY{n}{line}\PY{p}{)}\PY{o}{.}\PY{n}{strip}\PY{p}{(}\PY{p}{)}

        \PY{c+c1}{\PYZsh{} if line is empty, skip it}
        \PY{k}{if} \PY{n}{line} \PY{o+ow}{is} \PY{k+kc}{None} \PY{o+ow}{or} \PY{n}{line} \PY{o}{==} \PY{l+s+s2}{\PYZdq{}}\PY{l+s+s2}{\PYZdq{}} \PY{o+ow}{or} \PY{n+nb}{len}\PY{p}{(}\PY{n}{line}\PY{p}{)} \PY{o}{\PYZlt{}} \PY{l+m+mi}{1}\PY{p}{:} \PY{k}{continue}

        \PY{c+c1}{\PYZsh{} split the line (string) by seperator (\PYZdq{},\PYZdq{})}
        \PY{n}{line\PYZus{}data} \PY{o}{=} \PY{n}{line}\PY{o}{.}\PY{n}{split}\PY{p}{(}\PY{l+s+s1}{\PYZsq{}}\PY{l+s+s1}{,}\PY{l+s+s1}{\PYZsq{}}\PY{p}{)}

        \PY{c+c1}{\PYZsh{} column 7 (index: 6) needs to be removed}
        \PY{n}{line\PYZus{}data}\PY{o}{.}\PY{n}{pop}\PY{p}{(}\PY{l+m+mi}{6}\PY{p}{)}
        
        \PY{c+c1}{\PYZsh{} each value in the \PYZsq{}line\PYZus{}data\PYZsq{} needs to be converted to a}
        \PY{c+c1}{\PYZsh{} numerical type.}
        \PY{c+c1}{\PYZsh{} All of the columns in the data are of type \PYZdq{}int\PYZdq{}, except}
        \PY{c+c1}{\PYZsh{} for one, column[5], which is a float}
        \PY{k}{for} \PY{n}{data\PYZus{}item\PYZus{}index}\PY{p}{,} \PY{n}{data\PYZus{}item} \PY{o+ow}{in} \PY{n+nb}{enumerate}\PY{p}{(}\PY{n}{line\PYZus{}data}\PY{p}{)}\PY{p}{:}
            \PY{c+c1}{\PYZsh{} is current column index 5?}
            \PY{k}{if} \PY{n}{data\PYZus{}item\PYZus{}index} \PY{o+ow}{in} \PY{p}{[}\PY{l+m+mi}{5}\PY{p}{]}\PY{p}{:}
                \PY{c+c1}{\PYZsh{} cast to type float}
                \PY{n}{data\PYZus{}item} \PY{o}{=} \PY{n+nb}{float}\PY{p}{(}\PY{n}{data\PYZus{}item}\PY{p}{)}
            \PY{k}{else}\PY{p}{:}
                \PY{c+c1}{\PYZsh{} cast to type int}
                \PY{n}{data\PYZus{}item} \PY{o}{=} \PY{n+nb}{int}\PY{p}{(}\PY{n}{data\PYZus{}item}\PY{p}{)}
            \PY{c+c1}{\PYZsh{} update the data\PYZus{}item value after type casting}
            \PY{n}{line\PYZus{}data}\PY{p}{[}\PY{n}{data\PYZus{}item\PYZus{}index}\PY{p}{]} \PY{o}{=} \PY{n}{data\PYZus{}item}
        \PY{c+c1}{\PYZsh{} append row to (raw) data array}
        \PY{n}{data}\PY{o}{.}\PY{n}{append}\PY{p}{(}\PY{n}{line\PYZus{}data}\PY{p}{)}
    
    \PY{c+c1}{\PYZsh{} return the raw data}
    \PY{k}{return} \PY{n}{data}
\end{Verbatim}
\end{tcolorbox}

    \hypertarget{data-initialization}{%
\subsection*{Data Initialization}\label{data-initialization}}

    Get the \texttt{RAW\_DATA} (2-dimensional array) from the file contents:

    \begin{tcolorbox}[breakable, size=fbox, boxrule=1pt, pad at break*=1mm,colback=cellbackground, colframe=cellborder]
\prompt{In}{incolor}{8}{\boxspacing}
\begin{Verbatim}[commandchars=\\\{\}]
\PY{c+c1}{\PYZsh{} populate the raw data from the source file contents}
\PY{n}{RAW\PYZus{}DATA} \PY{o}{=} \PY{n}{getRawDataFromFile}\PY{p}{(}\PY{n}{DATA\PYZus{}FILENAME}\PY{p}{)}
\end{Verbatim}
\end{tcolorbox}

    \begin{Verbatim}[commandchars=\\\{\}]
Drive already mounted at /content/drive; to attempt to forcibly remount, call
drive.mount("/content/drive", force\_remount=True).
    \end{Verbatim}

    Initialize \texttt{DataFrame}:

    \begin{tcolorbox}[breakable, size=fbox, boxrule=1pt, pad at break*=1mm,colback=cellbackground, colframe=cellborder]
\prompt{In}{incolor}{9}{\boxspacing}
\begin{Verbatim}[commandchars=\\\{\}]
\PY{c+c1}{\PYZsh{} create a dataFrame from the RAW\PYZus{}DATA}
\PY{n}{df} \PY{o}{=} \PY{n}{pd}\PY{o}{.}\PY{n}{DataFrame}\PY{p}{(}\PY{n}{data}\PY{o}{=}\PY{n}{RAW\PYZus{}DATA}\PY{p}{,} \PY{n}{columns}\PY{o}{=}\PY{n}{DATA\PYZus{}COLUMN\PYZus{}NAMES}\PY{o}{.}\PY{n}{values}\PY{p}{(}\PY{p}{)}\PY{p}{)}
\end{Verbatim}
\end{tcolorbox}

    Initialize \texttt{numpy.ndarray}:

    \begin{tcolorbox}[breakable, size=fbox, boxrule=1pt, pad at break*=1mm,colback=cellbackground, colframe=cellborder]
\prompt{In}{incolor}{10}{\boxspacing}
\begin{Verbatim}[commandchars=\\\{\}]
\PY{c+c1}{\PYZsh{} create a numpy multi\PYZhy{}dim. array from the RAW\PYZus{}DATA}
\PY{n}{D} \PY{o}{=} \PY{n}{np}\PY{o}{.}\PY{n}{ndarray}\PY{p}{(}\PY{n}{shape}\PY{o}{=}\PY{p}{(}\PY{n+nb}{len}\PY{p}{(}\PY{n}{RAW\PYZus{}DATA}\PY{p}{)}\PY{p}{,} \PY{n+nb}{len}\PY{p}{(}\PY{n}{RAW\PYZus{}DATA}\PY{p}{[}\PY{l+m+mi}{0}\PY{p}{]}\PY{p}{)}\PY{p}{)}\PY{p}{,} \PY{n}{dtype}\PY{o}{=}\PY{n+nb}{float}\PY{p}{)}
\PY{c+c1}{\PYZsh{} populate the numpy data matrix (array)}
\PY{k}{for} \PY{n}{i}\PY{p}{,} \PY{n}{row} \PY{o+ow}{in} \PY{n+nb}{enumerate}\PY{p}{(}\PY{n}{RAW\PYZus{}DATA}\PY{p}{)}\PY{p}{:} \PY{n}{D}\PY{p}{[}\PY{n}{i}\PY{p}{]} \PY{o}{=} \PY{n}{np}\PY{o}{.}\PY{n}{array}\PY{p}{(}\PY{n}{row}\PY{p}{)}
\end{Verbatim}
\end{tcolorbox}

    \hypertarget{data-preview}{%
\subsubsection*{Data Preview}\label{data-preview}}

    The data is made up by 345 rows (instances) and 6 columns (attributes).

    \begin{tcolorbox}[breakable, size=fbox, boxrule=1pt, pad at break*=1mm,colback=cellbackground, colframe=cellborder]
\prompt{In}{incolor}{11}{\boxspacing}
\begin{Verbatim}[commandchars=\\\{\}]
\PY{n}{D}\PY{o}{.}\PY{n}{shape}
\end{Verbatim}
\end{tcolorbox}

            \begin{tcolorbox}[breakable, size=fbox, boxrule=.5pt, pad at break*=1mm, opacityfill=0]
\prompt{Out}{outcolor}{11}{\boxspacing}
\begin{Verbatim}[commandchars=\\\{\}]
(345, 6)
\end{Verbatim}
\end{tcolorbox}
        
    \begin{tcolorbox}[breakable, size=fbox, boxrule=1pt, pad at break*=1mm,colback=cellbackground, colframe=cellborder]
\prompt{In}{incolor}{12}{\boxspacing}
\begin{Verbatim}[commandchars=\\\{\}]
\PY{n}{D}
\end{Verbatim}
\end{tcolorbox}

            \begin{tcolorbox}[breakable, size=fbox, boxrule=.5pt, pad at break*=1mm, opacityfill=0]
\prompt{Out}{outcolor}{12}{\boxspacing}
\begin{Verbatim}[commandchars=\\\{\}]
array([[85., 92., 45., 27., 31.,  0.],
       [85., 64., 59., 32., 23.,  0.],
       [86., 54., 33., 16., 54.,  0.],
       {\ldots},
       [98., 77., 55., 35., 89., 15.],
       [91., 68., 27., 26., 14., 16.],
       [98., 99., 57., 45., 65., 20.]])
\end{Verbatim}
\end{tcolorbox}
        
    \begin{tcolorbox}[breakable, size=fbox, boxrule=1pt, pad at break*=1mm,colback=cellbackground, colframe=cellborder]
\prompt{In}{incolor}{13}{\boxspacing}
\begin{Verbatim}[commandchars=\\\{\}]
\PY{n}{df}\PY{o}{.}\PY{n}{describe}\PY{p}{(}\PY{p}{)}
\end{Verbatim}
\end{tcolorbox}

            \begin{tcolorbox}[breakable, size=fbox, boxrule=.5pt, pad at break*=1mm, opacityfill=0]
\prompt{Out}{outcolor}{13}{\boxspacing}
\begin{Verbatim}[commandchars=\\\{\}]
       Mean Corpuscular Volume  Alkaline Phosphotase  \textbackslash{}
count               345.000000            345.000000
mean                 90.159420             69.869565
std                   4.448096             18.347670
min                  65.000000             23.000000
25\%                  87.000000             57.000000
50\%                  90.000000             67.000000
75\%                  93.000000             80.000000
max                 103.000000            138.000000

       Alamine Aminotransferase  Aspartate Aminotransferase  \textbackslash{}
count                345.000000                  345.000000
mean                  30.405797                   24.643478
std                   19.512309                   10.064494
min                    4.000000                    5.000000
25\%                   19.000000                   19.000000
50\%                   26.000000                   23.000000
75\%                   34.000000                   27.000000
max                  155.000000                   82.000000

       Gamma-Glutamyl Transpeptidase  \textbackslash{}
count                     345.000000
mean                       38.284058
std                        39.254616
min                         5.000000
25\%                        15.000000
50\%                        25.000000
75\%                        46.000000
max                       297.000000

       Number of half-pint equivalents of alcoholic beverages drunk per day
count                                         345.000000
mean                                            3.455072
std                                             3.337835
min                                             0.000000
25\%                                             0.500000
50\%                                             3.000000
75\%                                             6.000000
max                                            20.000000
\end{Verbatim}
\end{tcolorbox}
        
    \hypertarget{estimated-covariance-matrix}{%
\subsubsection*{Estimated Covariance
Matrix}\label{estimated-covariance-matrix}}

    \begin{tcolorbox}[breakable, size=fbox, boxrule=1pt, pad at break*=1mm,colback=cellbackground, colframe=cellborder]
\prompt{In}{incolor}{14}{\boxspacing}
\begin{Verbatim}[commandchars=\\\{\}]
\PY{c+c1}{\PYZsh{} get the covariave matrix}
\PY{n}{Sigma} \PY{o}{=} \PY{n}{np}\PY{o}{.}\PY{n}{cov}\PY{p}{(}\PY{n}{D}\PY{o}{.}\PY{n}{T}\PY{p}{,} \PY{n}{ddof}\PY{o}{=}\PY{l+m+mi}{1}\PY{p}{)}
\PY{n}{Sigma}
\end{Verbatim}
\end{tcolorbox}

            \begin{tcolorbox}[breakable, size=fbox, boxrule=.5pt, pad at break*=1mm, opacityfill=0]
\prompt{Out}{outcolor}{14}{\boxspacing}
\begin{Verbatim}[commandchars=\\\{\}]
array([[  19.7855578 ,    3.59934277,   12.81884058,    8.40583923,
          38.81795585,    4.6423576 ],
       [   3.59934277,  336.63700708,   27.28273509,   26.97080384,
          95.89180991,    6.17290192],
       [  12.81884058,   27.28273509,  380.73019885,  145.25846815,
         385.60532524,   13.47177283],
       [   8.40583923,   26.97080384,  145.25846815,  101.29403438,
         208.45331143,    9.39236603],
       [  38.81795585,   95.89180991,  385.60532524,  208.45331143,
        1540.92489046,   44.70902005],
       [   4.6423576 ,    6.17290192,   13.47177283,    9.39236603,
          44.70902005,   11.14114425]])
\end{Verbatim}
\end{tcolorbox}
        
    \begin{tcolorbox}[breakable, size=fbox, boxrule=1pt, pad at break*=1mm,colback=cellbackground, colframe=cellborder]
\prompt{In}{incolor}{15}{\boxspacing}
\begin{Verbatim}[commandchars=\\\{\}]
\PY{c+c1}{\PYZsh{} print the covariance matrix}
\PY{n}{frmt\PYZus{}str} \PY{o}{=} \PY{l+s+s1}{\PYZsq{}}\PY{l+s+s1}{[ }\PY{l+s+s1}{\PYZsq{}} \PY{o}{+} \PY{l+s+s1}{\PYZsq{}}\PY{l+s+s1}{ }\PY{l+s+s1}{\PYZsq{}}\PY{o}{.}\PY{n}{join}\PY{p}{(}\PY{p}{[}\PY{l+s+s2}{\PYZdq{}}\PY{l+s+si}{\PYZob{}:\PYZlt{}7.2f\PYZcb{}}\PY{l+s+s2}{\PYZdq{}}\PY{p}{]} \PY{o}{*} \PY{n}{Sigma}\PY{o}{.}\PY{n}{shape}\PY{p}{[}\PY{l+m+mi}{1}\PY{p}{]}\PY{p}{)} \PY{o}{+} \PY{l+s+s1}{\PYZsq{}}\PY{l+s+s1}{ ]}\PY{l+s+s1}{\PYZsq{}}
\PY{k}{for} \PY{n}{row} \PY{o+ow}{in} \PY{n}{Sigma}\PY{p}{:} \PY{n+nb}{print}\PY{p}{(}\PY{n}{frmt\PYZus{}str}\PY{o}{.}\PY{n}{format}\PY{p}{(}\PY{o}{*}\PY{n}{row}\PY{p}{)}\PY{p}{)}
\end{Verbatim}
\end{tcolorbox}

    \begin{Verbatim}[commandchars=\\\{\}]
[ 19.79   3.60    12.82   8.41    38.82   4.64    ]
[ 3.60    336.64  27.28   26.97   95.89   6.17    ]
[ 12.82   27.28   380.73  145.26  385.61  13.47   ]
[ 8.41    26.97   145.26  101.29  208.45  9.39    ]
[ 38.82   95.89   385.61  208.45  1540.92 44.71   ]
[ 4.64    6.17    13.47   9.39    44.71   11.14   ]
    \end{Verbatim}

    \hypertarget{eigenvalues}{%
\subsubsection*{Eigenvalues}\label{eigenvalues}}

    \begin{tcolorbox}[breakable, size=fbox, boxrule=1pt, pad at break*=1mm,colback=cellbackground, colframe=cellborder]
\prompt{In}{incolor}{16}{\boxspacing}
\begin{Verbatim}[commandchars=\\\{\}]
\PY{n}{evalues}\PY{p}{,} \PY{n}{evectors} \PY{o}{=} \PY{n}{LA}\PY{o}{.}\PY{n}{eig}\PY{p}{(}\PY{n}{Sigma}\PY{p}{)}
\end{Verbatim}
\end{tcolorbox}

    \begin{tcolorbox}[breakable, size=fbox, boxrule=1pt, pad at break*=1mm,colback=cellbackground, colframe=cellborder]
\prompt{In}{incolor}{17}{\boxspacing}
\begin{Verbatim}[commandchars=\\\{\}]
\PY{n}{evalues}
\end{Verbatim}
\end{tcolorbox}

            \begin{tcolorbox}[breakable, size=fbox, boxrule=.5pt, pad at break*=1mm, opacityfill=0]
\prompt{Out}{outcolor}{17}{\boxspacing}
\begin{Verbatim}[commandchars=\\\{\}]
array([1704.23455772,    8.51379717,   19.59323592,   37.76391713,
        290.92117087,  329.48615401])
\end{Verbatim}
\end{tcolorbox}
        
    \begin{tcolorbox}[breakable, size=fbox, boxrule=1pt, pad at break*=1mm,colback=cellbackground, colframe=cellborder]
\prompt{In}{incolor}{18}{\boxspacing}
\begin{Verbatim}[commandchars=\\\{\}]
\PY{n}{evectors}
\end{Verbatim}
\end{tcolorbox}

            \begin{tcolorbox}[breakable, size=fbox, boxrule=.5pt, pad at break*=1mm, opacityfill=0]
\prompt{Out}{outcolor}{18}{\boxspacing}
\begin{Verbatim}[commandchars=\\\{\}]
array([[ 2.48915066e-02,  3.11252804e-01, -9.43765959e-01,
         1.08533543e-01,  4.94442234e-03,  1.98561179e-03],
       [ 7.49485205e-02,  6.73102408e-03,  1.32930005e-03,
        -4.14076557e-02, -3.81810281e-02,  9.95571826e-01],
       [ 2.92623798e-01, -1.30836279e-02, -3.21086702e-02,
        -3.49274184e-01,  8.89478140e-01, -2.31264856e-03],
       [ 1.50393856e-01,  5.10442210e-02,  1.28881103e-01,
         9.24576654e-01,  3.19083539e-01,  3.88528379e-02],
       [ 9.40591759e-01,  1.57223279e-02,  2.33472488e-02,
        -4.14320645e-02, -3.24855878e-01, -8.51287245e-02],
       [ 2.83421198e-02, -9.48710968e-01, -3.01857001e-01,
         8.91897708e-02,  8.68807215e-04,  8.42648015e-03]])
\end{Verbatim}
\end{tcolorbox}
        
    Sort eigenvalues (and corresponding vectors) in decending order.

    \begin{tcolorbox}[breakable, size=fbox, boxrule=1pt, pad at break*=1mm,colback=cellbackground, colframe=cellborder]
\prompt{In}{incolor}{19}{\boxspacing}
\begin{Verbatim}[commandchars=\\\{\}]
\PY{n}{idx} \PY{o}{=} \PY{n}{evalues}\PY{o}{.}\PY{n}{argsort}\PY{p}{(}\PY{p}{)}\PY{p}{[}\PY{p}{:}\PY{p}{:}\PY{o}{\PYZhy{}}\PY{l+m+mi}{1}\PY{p}{]}
\PY{n}{evalues} \PY{o}{=} \PY{n}{evalues}\PY{p}{[}\PY{n}{idx}\PY{p}{]}
\PY{n}{evectors} \PY{o}{=} \PY{n}{evectors}\PY{p}{[}\PY{p}{:}\PY{p}{,} \PY{n}{idx}\PY{p}{]}
\end{Verbatim}
\end{tcolorbox}

    \begin{tcolorbox}[breakable, size=fbox, boxrule=1pt, pad at break*=1mm,colback=cellbackground, colframe=cellborder]
\prompt{In}{incolor}{20}{\boxspacing}
\begin{Verbatim}[commandchars=\\\{\}]
\PY{n}{evalues}
\end{Verbatim}
\end{tcolorbox}

            \begin{tcolorbox}[breakable, size=fbox, boxrule=.5pt, pad at break*=1mm, opacityfill=0]
\prompt{Out}{outcolor}{20}{\boxspacing}
\begin{Verbatim}[commandchars=\\\{\}]
array([1704.23455772,  329.48615401,  290.92117087,   37.76391713,
         19.59323592,    8.51379717])
\end{Verbatim}
\end{tcolorbox}
        
    \begin{tcolorbox}[breakable, size=fbox, boxrule=1pt, pad at break*=1mm,colback=cellbackground, colframe=cellborder]
\prompt{In}{incolor}{21}{\boxspacing}
\begin{Verbatim}[commandchars=\\\{\}]
\PY{n}{evectors}
\end{Verbatim}
\end{tcolorbox}

            \begin{tcolorbox}[breakable, size=fbox, boxrule=.5pt, pad at break*=1mm, opacityfill=0]
\prompt{Out}{outcolor}{21}{\boxspacing}
\begin{Verbatim}[commandchars=\\\{\}]
array([[ 2.48915066e-02,  1.98561179e-03,  4.94442234e-03,
         1.08533543e-01, -9.43765959e-01,  3.11252804e-01],
       [ 7.49485205e-02,  9.95571826e-01, -3.81810281e-02,
        -4.14076557e-02,  1.32930005e-03,  6.73102408e-03],
       [ 2.92623798e-01, -2.31264856e-03,  8.89478140e-01,
        -3.49274184e-01, -3.21086702e-02, -1.30836279e-02],
       [ 1.50393856e-01,  3.88528379e-02,  3.19083539e-01,
         9.24576654e-01,  1.28881103e-01,  5.10442210e-02],
       [ 9.40591759e-01, -8.51287245e-02, -3.24855878e-01,
        -4.14320645e-02,  2.33472488e-02,  1.57223279e-02],
       [ 2.83421198e-02,  8.42648015e-03,  8.68807215e-04,
         8.91897708e-02, -3.01857001e-01, -9.48710968e-01]])
\end{Verbatim}
\end{tcolorbox}
        
    \hypertarget{total-variance}{%
\subsubsection*{Total Variance}\label{total-variance}}

    \begin{tcolorbox}[breakable, size=fbox, boxrule=1pt, pad at break*=1mm,colback=cellbackground, colframe=cellborder]
\prompt{In}{incolor}{22}{\boxspacing}
\begin{Verbatim}[commandchars=\\\{\}]
\PY{n}{total\PYZus{}var} \PY{o}{=} \PY{n+nb}{sum}\PY{p}{(}\PY{n}{np}\PY{o}{.}\PY{n}{diag}\PY{p}{(}\PY{n}{Sigma}\PY{p}{)}\PY{p}{)}
\PY{n}{total\PYZus{}var}
\end{Verbatim}
\end{tcolorbox}

            \begin{tcolorbox}[breakable, size=fbox, boxrule=.5pt, pad at break*=1mm, opacityfill=0]
\prompt{Out}{outcolor}{22}{\boxspacing}
\begin{Verbatim}[commandchars=\\\{\}]
2390.512832827772
\end{Verbatim}
\end{tcolorbox}
        
    \hypertarget{mean-centering}{%
\subsubsection*{Mean Centering}\label{mean-centering}}

    First, we must find the multi-dimensional mean:

    \begin{tcolorbox}[breakable, size=fbox, boxrule=1pt, pad at break*=1mm,colback=cellbackground, colframe=cellborder]
\prompt{In}{incolor}{23}{\boxspacing}
\begin{Verbatim}[commandchars=\\\{\}]
\PY{n}{multi\PYZus{}d\PYZus{}mean} \PY{o}{=} \PY{n}{np}\PY{o}{.}\PY{n}{mean}\PY{p}{(}\PY{n}{D}\PY{p}{,} \PY{n}{axis}\PY{o}{=}\PY{l+m+mi}{0}\PY{p}{)}
\PY{n}{multi\PYZus{}d\PYZus{}mean}
\end{Verbatim}
\end{tcolorbox}

            \begin{tcolorbox}[breakable, size=fbox, boxrule=.5pt, pad at break*=1mm, opacityfill=0]
\prompt{Out}{outcolor}{23}{\boxspacing}
\begin{Verbatim}[commandchars=\\\{\}]
array([90.15942029, 69.86956522, 30.4057971 , 24.64347826, 38.28405797,
        3.45507246])
\end{Verbatim}
\end{tcolorbox}
        
    Now we can center our data from the multi-dimensional mean:

    \begin{tcolorbox}[breakable, size=fbox, boxrule=1pt, pad at break*=1mm,colback=cellbackground, colframe=cellborder]
\prompt{In}{incolor}{24}{\boxspacing}
\begin{Verbatim}[commandchars=\\\{\}]
\PY{n}{centered\PYZus{}data} \PY{o}{=} \PY{n}{D} \PY{o}{\PYZhy{}} \PY{n}{multi\PYZus{}d\PYZus{}mean}
\PY{n}{centered\PYZus{}data}
\end{Verbatim}
\end{tcolorbox}

            \begin{tcolorbox}[breakable, size=fbox, boxrule=.5pt, pad at break*=1mm, opacityfill=0]
\prompt{Out}{outcolor}{24}{\boxspacing}
\begin{Verbatim}[commandchars=\\\{\}]
array([[ -5.15942029,  22.13043478,  14.5942029 ,   2.35652174,
         -7.28405797,  -3.45507246],
       [ -5.15942029,  -5.86956522,  28.5942029 ,   7.35652174,
        -15.28405797,  -3.45507246],
       [ -4.15942029, -15.86956522,   2.5942029 ,  -8.64347826,
         15.71594203,  -3.45507246],
       {\ldots},
       [  7.84057971,   7.13043478,  24.5942029 ,  10.35652174,
         50.71594203,  11.54492754],
       [  0.84057971,  -1.86956522,  -3.4057971 ,   1.35652174,
        -24.28405797,  12.54492754],
       [  7.84057971,  29.13043478,  26.5942029 ,  20.35652174,
         26.71594203,  16.54492754]])
\end{Verbatim}
\end{tcolorbox}
        
    \hypertarget{principle-component-analysis-pca}{%
\subsubsection*{Principle Component Analysis
(PCA)}\label{principle-component-analysis-pca}}

    \begin{tcolorbox}[breakable, size=fbox, boxrule=1pt, pad at break*=1mm,colback=cellbackground, colframe=cellborder]
\prompt{In}{incolor}{25}{\boxspacing}
\begin{Verbatim}[commandchars=\\\{\}]
\PY{n}{pca} \PY{o}{=} \PY{n}{PCA}\PY{p}{(}\PY{n}{n\PYZus{}components}\PY{o}{=}\PY{l+m+mi}{2}\PY{p}{)}
\end{Verbatim}
\end{tcolorbox}

    \begin{tcolorbox}[breakable, size=fbox, boxrule=1pt, pad at break*=1mm,colback=cellbackground, colframe=cellborder]
\prompt{In}{incolor}{26}{\boxspacing}
\begin{Verbatim}[commandchars=\\\{\}]
\PY{n}{pca\PYZus{}transformed\PYZus{}D} \PY{o}{=} \PY{n}{pca}\PY{o}{.}\PY{n}{fit\PYZus{}transform}\PY{p}{(}\PY{n}{D}\PY{p}{)}
\PY{n}{pca\PYZus{}transformed\PYZus{}centered\PYZus{}D} \PY{o}{=} \PY{n}{pca}\PY{o}{.}\PY{n}{fit\PYZus{}transform}\PY{p}{(}\PY{n}{centered\PYZus{}data}\PY{p}{)}
\end{Verbatim}
\end{tcolorbox}

    \hypertarget{data-visualization}{%
\subsection*{Data Visualization}\label{data-visualization}}

    \hypertarget{raw-data}{%
\subsubsection*{Raw Data}\label{raw-data}}

    Blood tests which are thought to be sensitive to liver disorder that
might arise from excessive alcohol consumption.

    \begin{tcolorbox}[breakable, size=fbox, boxrule=1pt, pad at break*=1mm,colback=cellbackground, colframe=cellborder]
\prompt{In}{incolor}{27}{\boxspacing}
\begin{Verbatim}[commandchars=\\\{\}]
\PY{n}{plt}\PY{o}{.}\PY{n}{figure}\PY{p}{(}\PY{n}{figsize}\PY{o}{=}\PY{p}{(}\PY{l+m+mi}{8}\PY{p}{,} \PY{l+m+mi}{10}\PY{p}{)}\PY{p}{,} \PY{n}{dpi}\PY{o}{=}\PY{l+m+mi}{100}\PY{p}{)}
\PY{k}{for} \PY{n}{i} \PY{o+ow}{in} \PY{n+nb}{range}\PY{p}{(}\PY{l+m+mi}{1}\PY{p}{,} \PY{l+m+mi}{6}\PY{p}{)}\PY{p}{:}
    \PY{n}{plt}\PY{o}{.}\PY{n}{scatter}\PY{p}{(}\PY{n}{D}\PY{p}{[}\PY{p}{:}\PY{p}{,}\PY{l+m+mi}{0}\PY{p}{]}\PY{p}{,} \PY{n}{D}\PY{p}{[}\PY{p}{:}\PY{p}{,}\PY{n}{i}\PY{p}{]}\PY{p}{,} \PY{n}{s}\PY{o}{=}\PY{l+m+mi}{1}\PY{p}{,} \PY{n}{c}\PY{o}{=}\PY{n}{dot\PYZus{}colors}\PY{p}{[}\PY{n}{i}\PY{o}{\PYZhy{}}\PY{l+m+mi}{1}\PY{p}{]}\PY{p}{,} \PY{n}{label}\PY{o}{=}\PY{n}{DATA\PYZus{}COL\PYZus{}VALS}\PY{p}{[}\PY{n}{i}\PY{p}{]}\PY{p}{)}
\PY{n}{plt}\PY{o}{.}\PY{n}{ylabel}\PY{p}{(}\PY{l+s+s1}{\PYZsq{}}\PY{l+s+s1}{Measurement}\PY{l+s+s1}{\PYZsq{}}\PY{p}{)}
\PY{n}{plt}\PY{o}{.}\PY{n}{xlabel}\PY{p}{(}\PY{n}{DATA\PYZus{}COL\PYZus{}VALS}\PY{p}{[}\PY{l+m+mi}{0}\PY{p}{]}\PY{p}{)}
\PY{n}{plt}\PY{o}{.}\PY{n}{legend}\PY{p}{(}\PY{n}{loc}\PY{o}{=}\PY{l+s+s1}{\PYZsq{}}\PY{l+s+s1}{upper left}\PY{l+s+s1}{\PYZsq{}}\PY{p}{)}
\PY{n}{plt}\PY{o}{.}\PY{n}{show}\PY{p}{(}\PY{p}{)}
\end{Verbatim}
\end{tcolorbox}

    \begin{center}
    \adjustimage{max size={0.9\linewidth}{0.9\paperheight}}{final-project_files/final-project_48_0.png}
    \end{center}
    { \hspace*{\fill} \\}
    
    \hypertarget{original-data-v.s.-mean-centered-data}{%
\subsubsection*{Original Data v.s. Mean Centered
Data}\label{original-data-v.s.-mean-centered-data}}

    \begin{tcolorbox}[breakable, size=fbox, boxrule=1pt, pad at break*=1mm,colback=cellbackground, colframe=cellborder]
\prompt{In}{incolor}{28}{\boxspacing}
\begin{Verbatim}[commandchars=\\\{\}]
\PY{n}{fig} \PY{o}{=} \PY{n}{plt}\PY{o}{.}\PY{n}{figure}\PY{p}{(}\PY{n}{figsize}\PY{o}{=}\PY{p}{(}\PY{l+m+mi}{8}\PY{p}{,} \PY{l+m+mi}{10}\PY{p}{)}\PY{p}{,} \PY{n}{dpi}\PY{o}{=}\PY{l+m+mi}{100}\PY{p}{)}
\PY{n}{ax1} \PY{o}{=} \PY{n}{fig}\PY{o}{.}\PY{n}{add\PYZus{}subplot}\PY{p}{(}\PY{l+m+mi}{111}\PY{p}{)}
\PY{n}{ax2} \PY{o}{=} \PY{n}{ax1}\PY{o}{.}\PY{n}{twinx}\PY{p}{(}\PY{p}{)}
\PY{k}{for} \PY{n}{i} \PY{o+ow}{in} \PY{n+nb}{range}\PY{p}{(}\PY{l+m+mi}{1}\PY{p}{,} \PY{l+m+mi}{6}\PY{p}{)}\PY{p}{:}
    \PY{n}{ax1}\PY{o}{.}\PY{n}{scatter}\PY{p}{(}\PY{n}{D}\PY{p}{[}\PY{p}{:}\PY{p}{,}\PY{l+m+mi}{0}\PY{p}{]}\PY{p}{,} \PY{n}{D}\PY{p}{[}\PY{p}{:}\PY{p}{,}\PY{n}{i}\PY{p}{]}\PY{p}{,} \PY{n}{s}\PY{o}{=}\PY{l+m+mi}{1}\PY{p}{,} \PY{n}{c}\PY{o}{=}\PY{l+s+s1}{\PYZsq{}}\PY{l+s+s1}{red}\PY{l+s+s1}{\PYZsq{}}\PY{p}{)}
\PY{k}{for} \PY{n}{i} \PY{o+ow}{in} \PY{n+nb}{range}\PY{p}{(}\PY{l+m+mi}{1}\PY{p}{,} \PY{l+m+mi}{6}\PY{p}{)}\PY{p}{:}
    \PY{n}{ax2}\PY{o}{.}\PY{n}{scatter}\PY{p}{(}\PY{n}{centered\PYZus{}data}\PY{p}{[}\PY{p}{:}\PY{p}{,}\PY{l+m+mi}{0}\PY{p}{]}\PY{p}{,} \PY{n}{centered\PYZus{}data}\PY{p}{[}\PY{p}{:}\PY{p}{,}\PY{n}{i}\PY{p}{]}\PY{p}{,} \PY{n}{s}\PY{o}{=}\PY{l+m+mi}{1}\PY{p}{,} \PY{n}{c}\PY{o}{=}\PY{l+s+s1}{\PYZsq{}}\PY{l+s+s1}{blue}\PY{l+s+s1}{\PYZsq{}}\PY{p}{)}
\PY{n}{ax1}\PY{o}{.}\PY{n}{legend}\PY{p}{(}\PY{p}{[}\PY{l+s+s1}{\PYZsq{}}\PY{l+s+s1}{original data}\PY{l+s+s1}{\PYZsq{}}\PY{p}{]}\PY{p}{,} \PY{n}{loc}\PY{o}{=}\PY{l+s+s1}{\PYZsq{}}\PY{l+s+s1}{upper right}\PY{l+s+s1}{\PYZsq{}}\PY{p}{)}
\PY{n}{ax2}\PY{o}{.}\PY{n}{legend}\PY{p}{(}\PY{p}{[}\PY{l+s+s1}{\PYZsq{}}\PY{l+s+s1}{centered data}\PY{l+s+s1}{\PYZsq{}}\PY{p}{]}\PY{p}{,} \PY{n}{loc}\PY{o}{=}\PY{l+s+s1}{\PYZsq{}}\PY{l+s+s1}{upper left}\PY{l+s+s1}{\PYZsq{}}\PY{p}{)}
\PY{n}{ax1}\PY{o}{.}\PY{n}{set\PYZus{}ylabel}\PY{p}{(}\PY{l+s+s1}{\PYZsq{}}\PY{l+s+s1}{Original Data Measurements}\PY{l+s+s1}{\PYZsq{}}\PY{p}{)}
\PY{n}{ax2}\PY{o}{.}\PY{n}{set\PYZus{}ylabel}\PY{p}{(}\PY{l+s+s1}{\PYZsq{}}\PY{l+s+s1}{Centered Data Measurements}\PY{l+s+s1}{\PYZsq{}}\PY{p}{)}
\PY{n}{ax1}\PY{o}{.}\PY{n}{set\PYZus{}xlabel}\PY{p}{(}\PY{n}{DATA\PYZus{}COL\PYZus{}VALS}\PY{p}{[}\PY{l+m+mi}{0}\PY{p}{]}\PY{p}{)}
\PY{n}{plt}\PY{o}{.}\PY{n}{show}\PY{p}{(}\PY{p}{)}
\end{Verbatim}
\end{tcolorbox}

    \begin{center}
    \adjustimage{max size={0.9\linewidth}{0.9\paperheight}}{final-project_files/final-project_50_0.png}
    \end{center}
    { \hspace*{\fill} \\}
    
    \hypertarget{centered-data-v.s.-projected-data}{%
\subsubsection*{Centered Data v.s. Projected
Data}\label{centered-data-v.s.-projected-data}}

    Let's project our data on to the largest eigenvector.

    \begin{tcolorbox}[breakable, size=fbox, boxrule=1pt, pad at break*=1mm,colback=cellbackground, colframe=cellborder]
\prompt{In}{incolor}{29}{\boxspacing}
\begin{Verbatim}[commandchars=\\\{\}]
\PY{n}{Sigma\PYZus{}centered} \PY{o}{=} \PY{n}{np}\PY{o}{.}\PY{n}{cov}\PY{p}{(}\PY{n}{pca\PYZus{}transformed\PYZus{}centered\PYZus{}D}\PY{o}{.}\PY{n}{T}\PY{p}{,} \PY{n}{ddof}\PY{o}{=}\PY{l+m+mi}{1}\PY{p}{)}

\PY{n}{evalues\PYZus{}centered}\PY{p}{,} \PY{n}{evectors\PYZus{}centered} \PY{o}{=} \PY{n}{LA}\PY{o}{.}\PY{n}{eig}\PY{p}{(}\PY{n}{Sigma\PYZus{}centered}\PY{p}{)}

\PY{c+c1}{\PYZsh{} sort}
\PY{n}{idx} \PY{o}{=} \PY{n}{evalues\PYZus{}centered}\PY{o}{.}\PY{n}{argsort}\PY{p}{(}\PY{p}{)}\PY{p}{[}\PY{p}{:}\PY{p}{:}\PY{o}{\PYZhy{}}\PY{l+m+mi}{1}\PY{p}{]}
\PY{n}{evalues\PYZus{}centered} \PY{o}{=} \PY{n}{evalues\PYZus{}centered}\PY{p}{[}\PY{n}{idx}\PY{p}{]}
\PY{n}{evectors\PYZus{}centered} \PY{o}{=} \PY{n}{evectors\PYZus{}centered}\PY{p}{[}\PY{p}{:}\PY{p}{,} \PY{n}{idx}\PY{p}{]}

\PY{n}{coords\PYZus{}along\PYZus{}eig0} \PY{o}{=} \PY{n}{evectors\PYZus{}centered}\PY{p}{[}\PY{p}{:}\PY{p}{,}\PY{l+m+mi}{0}\PY{p}{]}\PY{o}{.}\PY{n}{T}\PY{o}{.}\PY{n}{dot}\PY{p}{(}\PY{n}{pca\PYZus{}transformed\PYZus{}centered\PYZus{}D}\PY{o}{.}\PY{n}{T}\PY{p}{)}
\PY{n}{coords\PYZus{}along\PYZus{}eig1} \PY{o}{=} \PY{n}{evectors\PYZus{}centered}\PY{p}{[}\PY{p}{:}\PY{p}{,}\PY{l+m+mi}{1}\PY{p}{]}\PY{o}{.}\PY{n}{T}\PY{o}{.}\PY{n}{dot}\PY{p}{(}\PY{n}{pca\PYZus{}transformed\PYZus{}centered\PYZus{}D}\PY{o}{.}\PY{n}{T}\PY{p}{)}
\PY{n}{coords\PYZus{}along\PYZus{}eig0}
\end{Verbatim}
\end{tcolorbox}

            \begin{tcolorbox}[breakable, size=fbox, boxrule=.5pt, pad at break*=1mm, opacityfill=0]
\prompt{Out}{outcolor}{29}{\boxspacing}
\begin{Verbatim}[commandchars=\\\{\}]
array([-7.94013907e-01, -5.56860410e+00,  1.28506263e+01, -6.61028704e-01,
       -3.16517168e+01, -2.72797043e+01, -3.24661762e+01, -3.08199894e+01,
       -3.38109526e+01, -3.45647505e+01, -2.85428289e+01, -2.78498460e+01,
       -1.42506455e+01, -2.11624988e+01, -2.69263222e+01, -2.06214855e+01,
       -3.03984727e+01, -2.54608889e+01,  1.56401760e+01, -2.66157360e+01,
       -1.93977566e+01, -2.00776630e+01, -3.39468473e+01, -2.26958953e+01,
        6.18625502e+01, -1.52080423e+01, -2.30350924e+01, -3.06242149e+01,
       -2.14670900e+01, -2.39611948e+01, -2.15265037e+01,  2.39450873e+00,
       -8.74377620e+00,  2.18298389e+00, -2.94751566e+01,  8.41954814e+01,
       -1.06056017e+01, -3.58689708e+01, -8.24794406e+00, -3.46749835e+00,
        1.96572444e+01,  4.11234216e+00, -2.51380221e+01, -2.53193786e+01,
       -2.28387419e+01, -2.84698060e+01, -3.37818461e+01,  9.20970930e+00,
       -2.76135229e+01, -3.12557509e+00, -2.43480207e+01, -2.85873056e+01,
        5.91795101e+01, -3.11723763e+00, -1.28947144e+01, -2.39898569e+01,
       -2.07666550e+01, -2.32826200e+01, -3.38703651e+00, -1.60093160e+01,
       -1.03990614e+01, -2.52581126e+01, -3.51892997e+01, -3.19709530e+01,
       -1.10594350e+01, -1.96900438e+01, -3.17387692e+01, -1.73743289e+01,
       -2.56794384e+01, -2.81932345e+01,  1.63854503e+01, -2.85360694e+00,
       -2.93484245e+01, -1.46234448e+01, -2.58873164e+01, -7.29083906e-01,
        1.16301672e+02, -9.08084068e+00, -2.83051967e+01, -8.04630172e+00,
        2.83832954e+01,  1.03569424e+01, -1.41477256e+01, -1.61410305e+01,
        2.42846609e+02, -1.61410305e+01, -9.57840152e+00, -3.17739719e+01,
       -2.81040173e+01, -3.15990006e+01, -1.33056278e+01, -3.58649769e+01,
       -2.27294490e+00, -2.63260741e+01, -1.36310763e+01, -1.53614485e+01,
        1.52608902e+01,  6.00376631e+01, -4.84346965e+00, -1.41032602e+01,
       -1.57310722e+01,  3.41650803e+01, -2.92373608e+01, -2.65888457e+01,
       -2.34268121e+01, -2.29878703e+01, -8.13561148e+00, -1.68010763e+01,
       -3.66312398e+01,  1.26547626e+01, -2.50423768e+01, -1.60626511e+01,
       -2.37080007e+01, -2.11195252e+01,  1.29635072e+02, -3.59170128e+01,
       -2.24706213e+01, -2.81648875e+01, -1.66571795e+01, -2.76964173e+01,
        9.84917291e+00, -1.58736582e+01, -2.35225047e+01, -1.83380989e+01,
       -5.21300315e+00, -2.84349426e+01,  7.01217024e+00,  3.98415774e+01,
       -2.96942425e+01, -1.99339260e+01, -9.22284434e+00, -2.17677681e+01,
        7.21982565e+01,  1.30584793e+02, -3.09289297e+01, -2.87036818e+01,
       -1.88240683e+01, -1.91911334e+01,  6.75571679e+01, -3.09855758e+01,
       -2.20878390e+01, -1.19069955e+01, -2.37004572e+01, -1.23857098e+00,
       -2.98914338e+01,  1.26284222e+01,  9.86913782e+00,  4.75188874e+01,
       -2.09618560e+01, -2.37004572e+01,  6.20713708e+01, -1.27406522e+00,
       -2.18859075e+01, -3.11195147e+01,  3.43941317e+01,  2.82418243e+01,
        4.77868760e+01,  1.46139275e+01,  1.45090132e+01, -2.26821924e+01,
        2.64758524e+01, -2.49425819e+01, -8.70750716e+00,  3.81068131e+00,
       -2.84245991e+01, -1.04233978e+01,  5.65232182e+01,  8.83734968e+01,
        3.49742559e+01,  1.24759791e+01, -9.48512403e+00,  3.41344077e+01,
       -1.97299031e+01, -3.31627144e+01,  8.69212714e+01,  1.24759791e+01,
        2.46257040e+01, -2.80777293e+01,  1.73655039e+02,  1.07905400e+01,
        3.17153878e+01,  7.96647426e+01,  1.26425097e+01, -1.43188504e+01,
        3.36506764e+01,  2.84408340e+01,  6.34201428e+01,  1.01954878e+01,
        7.09081255e+01,  1.70810128e+02, -2.47632547e+01, -2.90799978e+01,
       -1.72682324e+00, -2.20002213e+01, -3.74839664e+01, -2.59067183e+01,
       -2.90954503e+01, -2.15012307e+01, -1.74055503e+01, -2.91022786e+01,
       -1.95560567e+01, -2.86208677e+01,  2.22451181e+01, -1.99099385e+01,
        9.38126360e+01, -6.06023867e+00, -2.76379176e+01, -2.02262355e+01,
       -1.64169712e+01, -3.34981419e+01, -1.64642682e+01, -1.01765438e+01,
       -7.46453247e+00, -2.97241294e+01, -2.85930086e+01, -2.02102714e+01,
       -3.31876259e+01,  1.85884839e+01, -2.18144432e+01,  9.26434637e+00,
       -2.61755367e+00, -1.79194272e+01, -2.81464522e+01, -2.81022157e+01,
       -2.99045277e+01, -2.57889930e+01,  2.00286155e+00,  2.97032682e+01,
        1.18884964e+01,  2.03997540e+01, -1.27592347e+01, -1.57567593e+01,
        1.55949575e+02, -2.86218448e+00,  2.57490567e+01, -8.15251311e+00,
       -1.96737203e+01, -9.53444531e+00, -3.08101738e+01, -1.99156007e+01,
       -2.23778522e+01, -3.07255197e+01, -2.56423174e+01, -5.37732266e+00,
       -2.33751907e+01, -2.57318869e+01, -2.25193348e+01, -3.04075154e+01,
       -2.80691506e+01,  4.75260521e+01, -2.84727865e+00,  7.18450897e+01,
       -2.55019474e+01,  1.22309054e+01,  5.23915762e+00, -1.52910340e+01,
       -2.44874860e+01, -1.74928034e+01, -2.19798612e+01, -7.61009312e+00,
        3.55174703e+01, -1.50206801e+01, -1.01976138e+01, -2.09646309e+01,
        3.11772659e+01, -1.07715560e+01, -3.18157857e+01, -8.98664570e+00,
       -2.68006294e+00, -2.32313106e+01, -2.97339128e+01, -1.77252253e+01,
       -3.13539472e+01, -2.58348536e+01, -1.58880656e+01, -5.20244477e+00,
        4.24192406e+01,  2.99829671e+01, -6.22141837e+00,  1.16201089e+01,
       -1.31915122e+01, -2.44249529e+01, -3.13146674e+01, -1.93006472e+01,
       -1.86349889e+01,  3.48949015e+01, -2.53143407e+01, -1.17550748e+01,
        6.43976833e+00,  1.70666800e+00, -2.98611344e+00, -2.79760520e+01,
       -2.37412567e+01,  6.55178864e+01,  2.38276831e+01,  1.06442486e+01,
       -8.18873375e+00,  1.69826668e+01, -2.19834009e+01,  1.20691815e+02,
       -1.28001380e+01, -3.31428015e+01, -3.07920310e+01,  2.90570255e+00,
        5.92353668e+00, -8.00506849e+00,  7.46488945e+01, -1.18166635e+01,
       -2.90473276e+01, -2.93237018e+00,  5.66614903e+01,  4.96158678e+01,
        5.93276227e+00, -2.10093392e+01, -2.28731489e+01,  1.67116629e+02,
        9.73040788e+01, -2.20878390e+01,  4.01985705e+00,  7.72639940e+00,
       -1.13847795e+01, -1.20970644e+00,  1.65311059e+02, -3.07436469e+01,
       -2.50254305e+01, -1.76704030e-01,  6.07976538e-01, -1.27709048e+01,
       -3.34606936e+00,  3.10228175e-01,  1.73116035e+02,  1.02788343e+01,
       -9.99387384e+00,  9.32390209e+01,  1.81740850e+01, -1.60813582e+00,
       -1.33034138e+01,  3.48377442e+00, -2.90276768e+01,  2.38699141e+01,
        2.01533164e+00,  1.64625157e+02,  5.75141905e+01, -2.33976376e+01,
        3.88197525e+01])
\end{Verbatim}
\end{tcolorbox}
        
    Observe that now, we now have a 1 dimensional representation of our data
that captures about 73\% of the total variance of the centered data set.

    \begin{tcolorbox}[breakable, size=fbox, boxrule=1pt, pad at break*=1mm,colback=cellbackground, colframe=cellborder]
\prompt{In}{incolor}{30}{\boxspacing}
\begin{Verbatim}[commandchars=\\\{\}]
\PY{n}{evalues\PYZus{}centered}\PY{p}{[}\PY{l+m+mi}{0}\PY{p}{]}\PY{o}{/}\PY{n+nb}{sum}\PY{p}{(}\PY{n}{np}\PY{o}{.}\PY{n}{diag}\PY{p}{(}\PY{n}{Sigma\PYZus{}centered}\PY{p}{)}\PY{p}{)}
\end{Verbatim}
\end{tcolorbox}

            \begin{tcolorbox}[breakable, size=fbox, boxrule=.5pt, pad at break*=1mm, opacityfill=0]
\prompt{Out}{outcolor}{30}{\boxspacing}
\begin{Verbatim}[commandchars=\\\{\}]
0.8379884946288922
\end{Verbatim}
\end{tcolorbox}
        
    \begin{tcolorbox}[breakable, size=fbox, boxrule=1pt, pad at break*=1mm,colback=cellbackground, colframe=cellborder]
\prompt{In}{incolor}{31}{\boxspacing}
\begin{Verbatim}[commandchars=\\\{\}]
\PY{n}{projected\PYZus{}along\PYZus{}eig0} \PY{o}{=} \PY{n}{np}\PY{o}{.}\PY{n}{zeros}\PY{p}{(}\PY{p}{(}\PY{n}{D}\PY{o}{.}\PY{n}{shape}\PY{p}{[}\PY{l+m+mi}{0}\PY{p}{]}\PY{p}{,}\PY{l+m+mi}{2}\PY{p}{)}\PY{p}{)}
\PY{k}{for} \PY{n}{i} \PY{o+ow}{in} \PY{n+nb}{range}\PY{p}{(}\PY{n+nb}{len}\PY{p}{(}\PY{n}{coords\PYZus{}along\PYZus{}eig0}\PY{p}{)}\PY{p}{)}\PY{p}{:}
    \PY{n}{projected\PYZus{}along\PYZus{}eig0}\PY{p}{[}\PY{n}{i}\PY{p}{,}\PY{p}{:}\PY{p}{]} \PY{o}{=} \PY{n}{coords\PYZus{}along\PYZus{}eig0}\PY{p}{[}\PY{n}{i}\PY{p}{]}\PY{o}{*}\PY{n}{evectors\PYZus{}centered}\PY{p}{[}\PY{p}{:}\PY{p}{,}\PY{l+m+mi}{0}\PY{p}{]}
\PY{n}{projected\PYZus{}along\PYZus{}eig1} \PY{o}{=} \PY{n}{np}\PY{o}{.}\PY{n}{zeros}\PY{p}{(}\PY{p}{(}\PY{n}{D}\PY{o}{.}\PY{n}{shape}\PY{p}{[}\PY{l+m+mi}{0}\PY{p}{]}\PY{p}{,}\PY{l+m+mi}{2}\PY{p}{)}\PY{p}{)}
\PY{k}{for} \PY{n}{i} \PY{o+ow}{in} \PY{n+nb}{range}\PY{p}{(}\PY{n+nb}{len}\PY{p}{(}\PY{n}{coords\PYZus{}along\PYZus{}eig1}\PY{p}{)}\PY{p}{)}\PY{p}{:}
    \PY{n}{projected\PYZus{}along\PYZus{}eig1}\PY{p}{[}\PY{n}{i}\PY{p}{,}\PY{p}{:}\PY{p}{]} \PY{o}{=} \PY{n}{coords\PYZus{}along\PYZus{}eig1}\PY{p}{[}\PY{n}{i}\PY{p}{]}\PY{o}{*}\PY{n}{evectors\PYZus{}centered}\PY{p}{[}\PY{p}{:}\PY{p}{,}\PY{l+m+mi}{1}\PY{p}{]}

\PY{n}{pjdev0} \PY{o}{=} \PY{n}{projected\PYZus{}along\PYZus{}eig0}
\PY{n}{pjdev1} \PY{o}{=} \PY{n}{projected\PYZus{}along\PYZus{}eig1}

\PY{n}{fig} \PY{o}{=} \PY{n}{plt}\PY{o}{.}\PY{n}{figure}\PY{p}{(}\PY{n}{figsize}\PY{o}{=}\PY{p}{(}\PY{l+m+mi}{8}\PY{p}{,} \PY{l+m+mi}{10}\PY{p}{)}\PY{p}{,} \PY{n}{dpi}\PY{o}{=}\PY{l+m+mi}{100}\PY{p}{)}
\PY{n}{ax} \PY{o}{=} \PY{n}{fig}\PY{o}{.}\PY{n}{add\PYZus{}subplot}\PY{p}{(}\PY{l+m+mi}{111}\PY{p}{)}
\PY{n}{ax}\PY{o}{.}\PY{n}{scatter}\PY{p}{(}\PY{n}{pca\PYZus{}transformed\PYZus{}centered\PYZus{}D}\PY{p}{[}\PY{p}{:}\PY{p}{,}\PY{l+m+mi}{0}\PY{p}{]}\PY{p}{,} \PY{n}{pca\PYZus{}transformed\PYZus{}centered\PYZus{}D}\PY{p}{[}\PY{p}{:}\PY{p}{,}\PY{l+m+mi}{1}\PY{p}{]}\PY{p}{,} \PY{n}{s}\PY{o}{=}\PY{l+m+mi}{10}\PY{p}{,} \PY{n}{c}\PY{o}{=}\PY{l+s+s1}{\PYZsq{}}\PY{l+s+s1}{b}\PY{l+s+s1}{\PYZsq{}}\PY{p}{,} \PY{n}{marker}\PY{o}{=}\PY{l+s+s1}{\PYZsq{}}\PY{l+s+s1}{s}\PY{l+s+s1}{\PYZsq{}}\PY{p}{,} \PY{n}{label}\PY{o}{=}\PY{l+s+s1}{\PYZsq{}}\PY{l+s+s1}{PCA Transformed Centered Data}\PY{l+s+s1}{\PYZsq{}}\PY{p}{)}
\PY{n}{ax}\PY{o}{.}\PY{n}{scatter}\PY{p}{(}\PY{n}{pjdev0}\PY{p}{[}\PY{p}{:}\PY{p}{,}\PY{l+m+mi}{0}\PY{p}{]}\PY{p}{,} \PY{n}{pjdev0}\PY{p}{[}\PY{p}{:}\PY{p}{,}\PY{l+m+mi}{1}\PY{p}{]}\PY{p}{,} \PY{n}{s}\PY{o}{=}\PY{l+m+mi}{10}\PY{p}{,} \PY{n}{c}\PY{o}{=}\PY{l+s+s1}{\PYZsq{}}\PY{l+s+s1}{r}\PY{l+s+s1}{\PYZsq{}}\PY{p}{,} \PY{n}{marker}\PY{o}{=}\PY{l+s+s1}{\PYZsq{}}\PY{l+s+s1}{x}\PY{l+s+s1}{\PYZsq{}}\PY{p}{,} \PY{n}{label}\PY{o}{=}\PY{l+s+s1}{\PYZsq{}}\PY{l+s+s1}{1\PYZhy{}dimensional pca\PYZhy{}transformed data}\PY{l+s+s1}{\PYZsq{}}\PY{p}{)}
\PY{n}{ax}\PY{o}{.}\PY{n}{scatter}\PY{p}{(}\PY{n}{pjdev1}\PY{p}{[}\PY{p}{:}\PY{p}{,}\PY{l+m+mi}{0}\PY{p}{]}\PY{p}{,} \PY{n}{pjdev1}\PY{p}{[}\PY{p}{:}\PY{p}{,}\PY{l+m+mi}{1}\PY{p}{]}\PY{p}{,} \PY{n}{s}\PY{o}{=}\PY{l+m+mi}{10}\PY{p}{,} \PY{n}{c}\PY{o}{=}\PY{l+s+s1}{\PYZsq{}}\PY{l+s+s1}{c}\PY{l+s+s1}{\PYZsq{}}\PY{p}{,} \PY{n}{marker}\PY{o}{=}\PY{l+s+s1}{\PYZsq{}}\PY{l+s+s1}{x}\PY{l+s+s1}{\PYZsq{}}\PY{p}{,} \PY{n}{label}\PY{o}{=}\PY{l+s+s1}{\PYZsq{}}\PY{l+s+s1}{1\PYZhy{}dimensional transformed data onto evec 1}\PY{l+s+s1}{\PYZsq{}}\PY{p}{)}
\PY{n}{plt}\PY{o}{.}\PY{n}{legend}\PY{p}{(}\PY{n}{loc}\PY{o}{=}\PY{l+s+s1}{\PYZsq{}}\PY{l+s+s1}{upper right}\PY{l+s+s1}{\PYZsq{}}\PY{p}{)}
\PY{n}{plt}\PY{o}{.}\PY{n}{title}\PY{p}{(}\PY{l+s+s1}{\PYZsq{}}\PY{l+s+s1}{Scatter plot of PCA\PYZhy{}transformed data and Projected Data}\PY{l+s+s1}{\PYZsq{}}\PY{p}{)}
\PY{n}{plt}\PY{o}{.}\PY{n}{show}\PY{p}{(}\PY{p}{)}
\end{Verbatim}
\end{tcolorbox}

    \begin{center}
    \adjustimage{max size={0.9\linewidth}{0.9\paperheight}}{final-project_files/final-project_56_0.png}
    \end{center}
    { \hspace*{\fill} \\}
    
    \hypertarget{part-3-report-40-points}{%
\section*{Part 3: Report (40 points)}\label{part-3-report-40-points}}

    \hypertarget{problem-statement}{%
\subsection*{Problem Statement}\label{problem-statement}}

    We where hoping to help identify key characteristics that can help show
potential signs of liver disorders.

    \hypertarget{data}{%
\subsection*{Data}\label{data}}

    This data set had 345 intstances and 7 attributes with no missing
values.

\begin{quote}
All of the attributes in our data were numerical.
\end{quote}

    \hypertarget{preprocessing-techniques}{%
\subsection*{Preprocessing Techniques}\label{preprocessing-techniques}}

    To preprocess our data, we used a few techiques. We mean centered our
data so that our data results would be standardized between the
techiques that require mean centering and those that do not.

    \hypertarget{data-mining-techniques}{%
\subsection*{Data Mining Techniques}\label{data-mining-techniques}}

    We used dimensionality reduction to identify which of the attributes has
the greatest effect on PCA. We used PCA as it very quickly can give us a
good estimation on which attributes hold the majority of the variance.
It can also show us how greatly an attribute effects the position of a
data point when a dimension is reduced.

    \hypertarget{analysis}{%
\subsection*{Analysis}\label{analysis}}

    Data was taken from
\href{https://archive.ics.uci.edu/ml/datasets/Liver+Disorders}{\emph{https://archive.ics.uci.edu/ml/datasets/Liver+Disorders}}

    \hypertarget{proportion-of-total-variance}{%
\subsubsection*{Proportion of Total
Variance}\label{proportion-of-total-variance}}

    In the direction of the largest eigenvalues, we capture 71\% of the
total variance. This shows us that while there is a large amount of data
in the major components, the other components may still contain valuable
information.

    \begin{tcolorbox}[breakable, size=fbox, boxrule=1pt, pad at break*=1mm,colback=cellbackground, colframe=cellborder]
\prompt{In}{incolor}{32}{\boxspacing}
\begin{Verbatim}[commandchars=\\\{\}]
\PY{n}{evalues}\PY{p}{[}\PY{l+m+mi}{0}\PY{p}{]}\PY{o}{/}\PY{n}{total\PYZus{}var}
\end{Verbatim}
\end{tcolorbox}

            \begin{tcolorbox}[breakable, size=fbox, boxrule=.5pt, pad at break*=1mm, opacityfill=0]
\prompt{Out}{outcolor}{32}{\boxspacing}
\begin{Verbatim}[commandchars=\\\{\}]
0.7129158791006928
\end{Verbatim}
\end{tcolorbox}
        
    \hypertarget{missing-values}{%
\subsubsection*{Missing Values}\label{missing-values}}

    There were no missing values from the data.

    \begin{tcolorbox}[breakable, size=fbox, boxrule=1pt, pad at break*=1mm,colback=cellbackground, colframe=cellborder]
\prompt{In}{incolor}{33}{\boxspacing}
\begin{Verbatim}[commandchars=\\\{\}]
\PY{n}{df}\PY{o}{.}\PY{n}{isnull}\PY{p}{(}\PY{p}{)}\PY{o}{.}\PY{n}{sum}\PY{p}{(}\PY{p}{)}
\end{Verbatim}
\end{tcolorbox}

            \begin{tcolorbox}[breakable, size=fbox, boxrule=.5pt, pad at break*=1mm, opacityfill=0]
\prompt{Out}{outcolor}{33}{\boxspacing}
\begin{Verbatim}[commandchars=\\\{\}]
Mean Corpuscular Volume                                                 0
Alkaline Phosphotase                                                    0
Alamine Aminotransferase                                                0
Aspartate Aminotransferase                                              0
Gamma-Glutamyl Transpeptidase                                           0
Number of half-pint equivalents of alcoholic beverages drunk per day    0
dtype: int64
\end{Verbatim}
\end{tcolorbox}
        
    \hypertarget{results}{%
\subsubsection*{Results}\label{results}}

    \hypertarget{time-constraints}{%
\subsubsection*{Time Constraints}\label{time-constraints}}

    We did not explore everything that we hoped to be able to when we
started the project. We where not able to run the same techiques on the
same dataset by separating the datasets into a test base and a real
data. We where also not able to run the k-means algorithm.

    \hypertarget{video-link}{%
\section*{Video Link}\label{video-link}}

    https://youtu.be/COPckqaGQb8

    \hypertarget{references}{%
\section*{References}\label{references}}

    Forsyth, Richard S. ``Liver Disorders Data Set.'' \emph{UCI Machine
Learning Repository}, 15 May 1990,
\href{https://archive.ics.uci.edu/ml/datasets/Liver+Disorders}{\emph{https://archive.ics.uci.edu/ml/datasets/Liver+Disorders}}.
Accessed 14 April 2022.


    % Add a bibliography block to the postdoc
    
    
    
\end{document}
